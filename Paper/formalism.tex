%\newcommand{\incolor}[1]{#1}    % Use to typeset in color
\newcommand{\incolor}[1]{}     % Use to typeset in black and white

% color framework

\newcommand{\judgecolor}{}
\newcommand{\typecolor}{}
\newcommand{\termcolor}{}
\newcommand{\Typecolor}{}
\newcommand{\Termcolor}{}

\newcommand{\uncolored}{
  \incolor{
    \renewcommand{\judgecolor}{}
    \renewcommand{\typecolor}{}
    \renewcommand{\termcolor}{}
    \renewcommand{\Typecolor}{}
    \renewcommand{\Termcolor}{}
  }
}

\newcommand{\colored}{
  \incolor{
    \renewcommand{\judgecolor}{\color{black}}
    \renewcommand{\typecolor}{\color{blue}}
    \renewcommand{\termcolor}{\color{red}}
    \renewcommand{\Typecolor}{\color{cyan}}
    \renewcommand{\Termcolor}{\color{magenta}}
  }
}

\newcommand{\tp}[1]{{\typecolor #1}}
\newcommand{\tm}[1]{{\termcolor #1}}

\newcommand{\reducedstrut}{\vrule width 0pt height .9\ht\strutbox depth .9\dp\strutbox\relax}
\definecolor{lightgray}{gray}{0.7}
\newcommand{\highlight}[1]{%
  \begingroup
  \setlength{\fboxsep}{0pt}%
  \colorbox{lightgray}{\reducedstrut\ensuremath{#1}\/}%
  \endgroup
}

\newcommand{\etal}{\emph{et~al.}}
\newcommand{\inference}[3]{\infer[\mathsf{#2}]{#3}{#1}}
\newcommand{\Inference}[3]{\infer=[\mathsf{#2}]{#3}{#1}}
\newcommand{\hole}{[\ ]}
\newcommand{\intro}{\mathcal{I}}
\newcommand{\elim}{\mathcal{E}}
\newcommand{\nv}{P}

%% Types
\newcommand{\typvar}[1]{#1}
\newcommand{\typzro}{\mathbf{0}}
\newcommand{\typone}{\mathbf{1}}
\newcommand{\typarr}[2]{#1\boldsymbol{\rightarrow}#2}
\newcommand{\typprd}[2]{#1\boldsymbol{\times}#2}
\newcommand{\typsum}[2]{#1\boldsymbol{+}#2}
\newcommand{\typrec}[2]{\boldsymbol{\mu}#1\boldsymbol{.}#2}

%% Expressions
\newcommand{\expvar}[1]{#1}
\newcommand{\expunt}{\boldsymbol{()}}
\newcommand{\expabs}[3]{\boldsymbol{\lambda}\,#1\>\boldsymbol{.}\,#3}
\newcommand{\expapp}[2]{#1\ #2}
\newcommand{\expshr}[3]{\mathbf{let}\ #1\boldsymbol{=}#2\ \mathbf{in}\ #3}
\newcommand{\expshrind}[3]{\begin{array}[t]{@{}l@{}}\mathbf{let}\ #1\boldsymbol{=}#2\\ \mathbf{in}\ \ #3\end{array}}
\newcommand{\expprd}[2]{\boldsymbol{(}#1\ \boldsymbol{,}\ #2\boldsymbol{)}}
\newcommand{\expfst}[1]{\mathbf{fst}\ #1}
\newcommand{\expsnd}[1]{\mathbf{snd}\ #1}
\newcommand{\explft}[2]{\mathbf{inl}\ #1}
\newcommand{\exprgt}[2]{\mathbf{inr}\ #2}
\newcommand{\expcas}[5]{\mathbf{case}\ #1\ \mathbf{of}\ \boldsymbol{\{}\mathbf{inl}\ #2.\ #3\ \boldsymbol{;}\ \mathbf{inr}\ #4.\ #5\boldsymbol{\}}}
\newcommand{\expcasind}[5]{\begin{array}[t]{@{}l@{}}\mathbf{case}\ #1\ \mathbf{of}\\[-1pt] \ \ \boldsymbol{\{}\mathbf{inl}\ #2.\ #3\ \boldsymbol{;}\ \mathbf{inr}\ #4.\ #5\boldsymbol{\}}\end{array}}
\newcommand{\expcasindind}[5]{\begin{array}[t]{@{}l@{}}\mathbf{case}\ #1\ \mathbf{of}\\[-1pt] \ \ \ \ \boldsymbol{\{}\mathbf{inl}\ #2.\ #3\ \\[-1pt] \ \ \boldsymbol{;}\ \mathbf{inr}\ #4.\ #5\boldsymbol{\}}\end{array}}

%% Environment
\newcommand{\envemp}{\tp{\boldsymbol{\emptyset}}}
\newcommand{\envcon}[2]{\tp{#1,}\ #2}
\newcommand{\env}{\tp{\Gamma}}
\newcommand{\typing}[2]{\tm{#1:\ }\tp{#2}}

\newcommand{\typenvcon}[2]{\tp{\Gamma,}\ \typing{#1}{#2}}
\newcommand{\sbs}[3]{#1[#2:=#3]}
\newcommand{\fv}[1]{\txt{FV}\,(#1)}
\newcommand{\fresh}[1]{\txt{Fresh}\,(#1)}
\newcommand{\txt}[1]{\text{\textit{#1}}}
\newcommand{\rewrite}[3]{#1 \mapsto_{#2} #3}
\newcommand{\reduce}[3]{#1 \rightarrow_{#2} #3}
\newcommand{\reducestar}[3]{#1 \twoheadrightarrow_{#2} #3}
\newcommand{\valuep}[1]{\txt{Value}\,(#1)}
\newcommand{\cnd}[1]{\begin{array}[t]{@{}l@{}}\txt{if}\ #1\end{array}}
\newcommand{\subformulas}[1]{\txt{Subformulas}\,(#1)}
\newcommand{\psubformulas}[1]{\txt{ProperSubformulas}\,(#1)}
\newcommand{\subterm}[1]{\txt{Subterm}\,(#1)}
\newcommand{\norm}[1]{\txt{Norm}\,(#1)}

\newcommand{\na}{L^{\text{\textcrlambda}}}


\newcommand{\figterm}{
\begin{figure*}[t]
\[\uncolored
\begin{array}{l@{\quad}rcl}
\text{Types} & A,B,C & ::=&
% \typone             %   & \textrm{unit}      \\
  \iota         \mid  %   & \textrm{base type} \\
  \typarr{A}{B} \mid  %   & \textrm{function}  \\
  \typprd{A}{B} \mid  %   & \textrm{product}   \\
  \typsum{A}{B}       %   & \textrm{sum}
\\[1ex]
\text{Terms} & L,M,N & ::= &
% \expunt                \mid   % \textrm{unit}              \\
  \expvar{x}             \mid   % \textrm{variable}          \\
  \expabs{x}{A}{N}       \mid   % \textrm{abstraction}       \\
  \expapp{L}{M}          \mid   % \textrm{application}       \\
  \expshr{x}{M}{N}       \mid   % \textrm{sharing}           \\
  \expprd{M}{N}          \mid   % \textrm{product}           \\
  \expfst{L}             \mid   % \textrm{projection-first}  \\
  \expsnd{L}             \\&&&  % \textrm{projection-second} \\
  \explft{M}{\tm{B}}     \mid   % \textrm{injection-left}    \\
  \exprgt{\tm{A}}{N}     \mid   % \textrm{injection-right}   \\
  \expcas{L}{x}{M}{y}{N}        % \textrm{case}
\\[1ex]
\text{Values} & V,W & ::= &
% \expunt\               \mid
  \expvar{x}             \mid
  \expabs{x}{A}{N}       \mid
  \expprd{V}{W}          \mid
  \explft{V}{B}          \mid
  \exprgt{A}{W}
\end{array}
\]
\caption{Types, Terms, and Values}
\label{fig:term}
\end{figure*}
}

\newcommand{\figtyping}{
\begin{figure*}[h]
\[\colored
\begin{array}{@{}ll@{}}
\fbox{$\env \vdash \typing{M}{A}$}
\\~\\
\inference
{\typing{x}{A} \in \env}
{\mathbf{Ax}}
{
  \env \vdash \typing{x}{A}
}
&
\inference
{}
{\typone}
{
   \env \vdash \typing{\expunt}{\typone}
}
\\~\\
\inference
{
  \typenvcon{x}{A} \vdash \typing{N}{B}
}
{{\to}\intro}
{
  \env \vdash \typing{\expabs{x}{A}{N}}{\typarr{A}{B}}
}
&
\inference
{
  \env \vdash \typing{L}{\typarr{A}{B}}
& \env \vdash \typing{M}{A}
}
{{\to}\elim}
{
  \env \vdash \typing{\expapp{L}{M}}{B}
}
\\~\\
\inference
{
  \env \vdash \typing{M}{A}
  &
  \typenvcon{x}{A} \vdash \typing{N}{B}
}
{\mathbf{let}}
{
  \env \vdash \typing{\expshr{x}{M}{N}}{B}
}
&
\inference
{
  \env \vdash \typing{M}{A}
  &
  \env \vdash \typing{N}{B}
}
{{\times}\intro}
{
  \env \vdash \typing{\expprd{M}{N}}{\typprd{A}{B}}
}
\\~\\
\inference
{
  \env \vdash \typing{L}{\typprd{A}{B}}
}
{{\times}\elim_1}
{
  \env \vdash \typing{\expfst{L}}{A}
}
&
\inference
{
  \env \vdash \typing{L}{\typprd{A}{B}}
}
{{\times}\elim_2}
{
  \env \vdash \typing{\expsnd{L}}{B}
}
\\~\\
\inference
{
  \env \vdash \typing{M}{A}
}
{{+}\intro_1}
{
  \env \vdash \typing{\explft{M}{B}}{\typsum{A}{B}}
}
&
\inference
{
  \env \vdash \typing{N}{B}
}
{{+}\intro_2}
{
  \env \vdash \typing{\exprgt{A}{N}}{\typsum{A}{B}}
}
\\~\\
\inference
{
  \env \vdash \typing{L}{\typsum{A}{B}}
&
  \typenvcon{x}{A} \vdash \typing{M}{C}
&
  \typenvcon{y}{B} \vdash \typing{N}{C}
}
{{+}\elim}
{
  \env \vdash \typing{\expcas{L}{x}{M}{y}{N}}{C}
}
\end{array}
\]
\caption{Typing Rules}
\label{fig:typing}
\end{figure*}
}

\newcommand{\fignf}{
\begin{figure*}[t]
\[\uncolored
\begin{array}{l@{\quad}rcl}
\text{Neutral Forms} & Q & ::= &
   \expapp{x}{W}   \mid
   \expapp{Q}{W}   \mid
   \expfst{x}      \mid
   \expsnd{x}      
\\[1ex]
\text{Value Forms} & V,W & ::= &
% \expunt{}          \mid
   \expvar{x}        \mid
   \expabs{x}{A}{N}  \mid
   \expprd{V}{W}     \mid
   \explft{V}{B}     \mid 
   \exprgt{A}{W}
\\[1ex]
\text{Normal Forms} & N,M & ::= &
   Q  \mid  V  \mid \expcas{z}{x}{N}{y}{M} \mid \expshr{x}{Q}{N}
\\
\end{array}
\]
\caption{Normal Forms}
\label{fig:nf}
\end{figure*}
}

\newcommand{\fignorm}{
\begin{figure*}[t]
\[\uncolored
\begin{array}[t]{@{}lllll@{}}
\multicolumn{5}{@{}c@{}}{\txt{Phase 1}}\\[-9pt]
\multicolumn{5}{@{}c@{}}{\rule{120pt}{1pt}}\\[1pt]
(\eta_{\rightarrow})
& L
& \ \rewrite{}{1}{}\ \
& \expabs{x}{A}{\expapp{L}{x}}
& %\cnd{%% \neg\valuep{M},\ and\\\ \ \ \fresh{x}}
\\[0pt]
\multicolumn{5}{@{}l@{}}{\txt{\ \ \ \ where\ \ \ }\env \vdash \typing{L}{\typarr{A}{B}},\ L \neq \expabs{y}{A}{N}\txt{, and }\fresh{x}}
\\[10pt]

\multicolumn{5}{@{}c@{}}{\txt{Phase 2}}\\[-9pt]
\multicolumn{5}{@{}c@{}}{\rule{120pt}{1pt}}\\[1pt]

\multicolumn{5}{@{}l@{}}{F \mathbin{\ ::=\ }
    \expapp{M}{\hole}\
 |\ \expprd{\hole}{N}\
 |\ \expprd{V}{\hole}\
 |\ \expfst{\hole}\
 |\ \expsnd{\hole}\
 |\ \explft{\hole}{B}\
 |\ \exprgt{A}{\hole}\
 |\ \expcasind{\hole}{x}{M}{y}{N}}\\[1pt]

(\txt{let}_{\rightarrow\times+})
& F[\nv]
& \ \rewrite{}{2}{}\ \
& \expshr{x}{\nv}{F[x]}
& \cnd{%% \neg\valuep{M},\ and\\\ \ \
       \fresh{x}} \\[8pt]

\multicolumn{5}{@{}c@{}}{\txt{Phase 3}}\\[-9pt]
\multicolumn{5}{@{}c@{}}{\rule{120pt}{1pt}}\\[1pt]

\multicolumn{5}{@{}l@{}}{G \mathbin{\ ::=\ }
    \expapp{\hole}{V}\
 |\ \expshr{x}{\hole}{N}}\\[1pt]

(\kappa_{\txt{let}})
& G[\expshr{x}{\nv}{N}]
& \ \rewrite{}{3}{}\ \
& \expshr{x}{\nv}{G[N]}
& \cnd{x \notin \fv{G}}\\[0pt]

(\kappa_{\txt{case}})
& G[\expcasind{z}{x}{M}{y}{N}]
& \ \rewrite{}{3}{}\ \
& \expcasind{z}{x}{G[M]}{y}{G[N]}
& \cnd{x \notin \fv{G},\ \txt{and}\\\ \ \ \, y \notin \fv{G}} \\[20pt]

%% \multicolumn{5}{@{}l@{}}{H \mathbin{\ ::=\ } \expshr{x}{M}{\hole}\ |\ \expcas{z}{x}{\hole}{y}{N}\ |\ \expcas{z}{x}{M}{y}{\hole}}\\[1pt]

%% (\kappa_{\lambda})
%% & H[\expabs{x}{A}{N}]
%% & \ \rewrite{}{}{}\ \
%% & \expabs{x}{A}{H[N]}
%% & \cnd{x \notin \fv{H}} \\[8pt]


(\beta_{\rightarrow})
& \expapp{(\expabs{x}{A}{N})}{V}
& \ \rewrite{}{3}{}\ \
& \sbs{N}{x}{V}
& \\[0pt]

(\beta_{\times_1})
& \expfst{\expprd{V}{W}}
& \ \rewrite{}{3}{}\ \
& V
& \\[0pt]

(\beta_{\times_2})
& \expsnd{\expprd{V}{W}}
& \ \rewrite{}{3}{}\ \
& W
& \\[0pt]

(\beta_{+_1})
& \expcasind{(\explft{V}{B})}{x}{M}{y}{N}
& \ \rewrite{}{3}{}\ \
& \sbs{M}{x}{V}
& \\[0pt]

(\beta_{+_2})
& \expcasind{(\exprgt{A}{W})}{x}{M}{y}{N}
& \ \rewrite{}{3}{}\ \
& \sbs{N}{y}{W}
& \\[0pt]

(\beta_{\txt{let}})
& \expshr{x}{V}{N}
& \ \rewrite{}{3}{}\ \
& \sbs{N}{x}{V}
& \\[8pt]

\multicolumn{5}{@{}c@{}}{\txt{Phase 4}}\\[-9pt]
\multicolumn{5}{@{}c@{}}{\rule{120pt}{1pt}}\\[1pt]

(\txt{need})
& \expshr{x}{\nv}{N}
& \ \rewrite{}{4}{}\ \
& N %%\sbs{N}{x}{\nv}
& \cnd{%% \neg\valuep{M},\ and\\\ \ \
       %%Count(x,N) < 2
       x \notin \fv{N}}\\[0pt]
\end{array}
\]
\caption{Normalisation by rewrite}
\label{fig:norm}
\end{figure*}
}

%%%%%%%%%%%%%%%%%%%%%%%%%%%%%%%%%%%%%%%%%%%%%%%%%%%%%%%%%%%%%%%%%%%%%%%%

\figterm
\fignf
\fignorm

%\todo{Removed unit type. Either remove unit type or add empty type.}

This section introduces a set of rewriting rules that normalise terms
to ensure the subformula property. Care is taken to ensure sharing is
preserved. We present rules suitable for both call-by-need and
call-by-value reduction.

We work with simply-typed lambda calculus, which is adequate for
our purpose. The only polymorphism in our examples corresponds to
instantiating variables (such as $\mathit{while}$) at different types,
which is easily modelled by a family of variables.

Types, terms, and values are presented as in Figure~\ref{fig:term}.
Let $A$, $B$, $C$ range over types, which include base types
$\iota$, function, product, and sum types. 
Let $L$, $M$, $N$ range over terms, and $x$, $y$, $z$ range
over variables. We follow the usual convention that terms are
equivalent up to variable renaming. Write $\fv{N}$ for the set of
free variables of $N$, and $\sbs{N}{x}{M}$ for
substitution of $M$ for $x$ in $N$. Types correspond to
propositions and terms to proofs in minimal propositional logic.
Let $V$, $W$ range over values, and
$\nv$ range over non-values (that is, any term that is not a value).

Let $\Gamma$ range over environments, which are sets of pairs
of variables with types $x:A$. Write $\Gamma \vdash M:A$ to indicate
that term $M$ has type $A$ under environment $\Gamma$.
Typing rules are as normal, and omitted to save space.

The normal forms of interest are presented in Figure~\ref{fig:nf}.
We reuse $L,M,N$ to ranges over terms in normal form, and
$V,W$ to range over value in normal form, and $Q$ to range of
neutral forms.

Rewrite rules for normalisation are presented in Figure~\ref{fig:norm},
and broken into four phases. Write $M \mapsto_i N$ to indicate
that $M$ reduces to $N$ in phase $i$.  Let $F$ and $G$ range over two
different forms of evaluation context used in Phases~2 and~3
respectively. Write $\fv{F}$ for the set of free variables of $F$, and
similarly for $G$.

Reductions are closed under transitivity and compatible closure.
Overall normalisation consists of applying reductions of Phase~1 until
no more apply, then Phase~2 similarly, and so on through Phase~4.
Phase~1 introduces $\eta$-expansions; some of these may be eliminated by
Phase~3 but that is not a problem. Phase~2 names subterms that are
not values, similar to A normal form \citep{a-normal-form} or
reductions (let.1) and (let.2) in Moggi's metalanguage for monads
\citep{moggi}. Phase~3 performs standard beta and commuting reductions.
Phase~4 ``garbage collects'' unused terms, similar to the
call-by-need lambda calculus \citep{call-by-need}.
Phase~4 may be omitted if the intended semantics of the target
language is call-by-value rather than call-by-need.

%%  \todo{Following sentence to be deleted and replaced by suitable
%%  formalism, e.g., definition of evaluation contexts $E$.}
%%  One-step reduction relation for phase 1 has an
%%  extra condition such that a redex should not be immediately at the
%%  left-hand side of an application, i.e. $E \neq
%%  {E}'[\expapp{\hole}{M}]$ for the compatible contexts $E$ and $E'$.

%%  Reduction relation, denoted as $\reducestar{}{i}{}$ for phase $i$, is
%%  reflexive transitive closure of the corresponding one-step reduction
%%  relation $\reduce{}{i}{}$. The overall normalisation, denoted as
%%  $\reducestar{}{}{}$, is composition of the reduction for all four
%%  phases,
%%  i.e. $\reducestar{}{4}{}\circ\reducestar{}{3}{}\circ\reducestar{}{2}{}\circ\reducestar{}{1}{}$.

%%  Typed terms and reduction satisfy the usual progress and preservation property.
%%  \todo{Fix! Progress requires adjustment for phases}
%%  
%%  \begin{proposition}[Progress and preservation]
%%  \label{prop_preservation}
%%  If $\Gamma \vdash M:A$ then either $M$ is a value, or there exists a
%%  term $N$ such that $M \mapsto_i N$ and $\Gamma \vdash N:A$.
%%  \end{proposition}
%%  
Our grammar characterises normal forms precisely.

\begin{proposition}[Normal Form Syntax]
\label{prop_normal}
An expression $N$ matches the syntax of normal forms in Figure~\ref{fig:nf}
if and only if it is in normal form with regard to the reduction rules of
Figure~\ref{fig:norm}.
\end{proposition}

The \emph{subformulas} of a type are the type itself and its components;
for instance, the subformulas of $A \to B$ are $A \to B$ itself
and the subformulas of $A$ and $B$. The \emph{proper subformulas} of a
type are all its subformulas other than the type itself.
Terms in normal form satisfy the subformula property.

\begin{proposition}[Subformula property]
\label{prop_subformula}
If $\Gamma \vdash M:A$ and the normal form of $M$ is $N$
by the reduction rules of Figure~\ref{fig:norm}, then
$\Gamma \vdash N:A$ and every subterm of $N$ has a type
that is either a subformula of $A$ or a subformula of a type in $\Gamma$.
Further, every subterm other than $N$ itself and free variables of $N$
has a type that is a proper subformula of $A$ or a proper subformula of a type in $\Gamma$.
\end{proposition}




