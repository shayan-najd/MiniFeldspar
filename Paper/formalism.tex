\newcommand{\incolor}[1]{#1}    % Use to typeset in color
%\newcommand{\incolor}[1]{}     % Use to typeset in black and white

% color framework

\newcommand{\judgecolor}{}
\newcommand{\typecolor}{}
\newcommand{\termcolor}{}
\newcommand{\Typecolor}{}
\newcommand{\Termcolor}{}

\newcommand{\uncolored}{
  \incolor{
    \renewcommand{\judgecolor}{}
    \renewcommand{\typecolor}{}
    \renewcommand{\termcolor}{}
    \renewcommand{\Typecolor}{}
    \renewcommand{\Termcolor}{}
  }
}

\newcommand{\colored}{
  \incolor{
    \renewcommand{\judgecolor}{\color{black}}
    \renewcommand{\typecolor}{\color{blue}}
    \renewcommand{\termcolor}{\color{red}}
    \renewcommand{\Typecolor}{\color{cyan}}
    \renewcommand{\Termcolor}{\color{magenta}}
  }
}

\newcommand{\tp}[1]{{\typecolor #1}}
\newcommand{\tm}[1]{{\termcolor #1}}

\newcommand{\reducedstrut}{\vrule width 0pt height .9\ht\strutbox depth .9\dp\strutbox\relax}
\definecolor{lightgray}{gray}{0.7}
\newcommand{\highlight}[1]{%
  \begingroup
  \setlength{\fboxsep}{0pt}%
  \colorbox{lightgray}{\reducedstrut\ensuremath{#1}\/}%
  \endgroup
}

\newcommand{\etal}{\emph{et~al.}}
\newcommand{\inference}[3]{\infer[\mathsf{#2}]{#3}{#1}}
\newcommand{\Inference}[3]{\infer=[\mathsf{#2}]{#3}{#1}}
\newcommand{\hole}{[\ ]}
\newcommand{\intro}{\mathcal{I}}
\newcommand{\elim}{\mathcal{E}}

%% Types
\newcommand{\typvar}[1]{#1}
\newcommand{\typzro}{\mathbf{0}}
\newcommand{\typone}{\mathbf{1}}
\newcommand{\typarr}[2]{#1\boldsymbol{\rightarrow}#2}
\newcommand{\typprd}[2]{#1\boldsymbol{\times}#2}
\newcommand{\typsum}[2]{#1\boldsymbol{+}#2}
\newcommand{\typrec}[2]{\boldsymbol{\mu}#1\boldsymbol{.}#2}

%% Expressions
\newcommand{\expvar}[1]{#1}
\newcommand{\expunt}{\boldsymbol{()}}
\newcommand{\expabs}[3]{\boldsymbol{\lambda}\,#1\>\boldsymbol{.}\,#3}
\newcommand{\expapp}[2]{#1\ #2}
\newcommand{\expshr}[3]{\mathbf{let}\ #1\boldsymbol{=}#2\ \mathbf{in}\ #3}
\newcommand{\expshrind}[3]{\begin{array}[t]{@{}l@{}}\mathbf{let}\ #1\boldsymbol{=}#2\\ \mathbf{in}\ \ #3\end{array}}
\newcommand{\expprd}[2]{\boldsymbol{(}#1\ \boldsymbol{,}\ #2\boldsymbol{)}}
\newcommand{\expfst}[1]{\mathbf{fst}\ #1}
\newcommand{\expsnd}[1]{\mathbf{snd}\ #1}
\newcommand{\explft}[2]{\mathbf{inl}\ #1}
\newcommand{\exprgt}[2]{\mathbf{inr}\ #2}
\newcommand{\expcas}[5]{\mathbf{case}\ #1\ \mathbf{of}\ \boldsymbol{\{}\mathbf{inl}\ #2.\ #3\ \boldsymbol{;}\ \mathbf{inr}\ #4.\ #5\boldsymbol{\}}}
\newcommand{\expcasind}[5]{\begin{array}[t]{@{}l@{}}\mathbf{case}\ #1\ \mathbf{of}\\[-1pt] \ \ \boldsymbol{\{}\mathbf{inl}\ #2.\ #3\ \boldsymbol{;}\ \mathbf{inr}\ #4.\ #5\boldsymbol{\}}\end{array}}
\newcommand{\expcasindind}[5]{\begin{array}[t]{@{}l@{}}\mathbf{case}\ #1\ \mathbf{of}\\[-1pt] \ \ \ \ \boldsymbol{\{}\mathbf{inl}\ #2.\ #3\ \\[-1pt] \ \ \boldsymbol{;}\ \mathbf{inr}\ #4.\ #5\boldsymbol{\}}\end{array}}

%% Environment
\newcommand{\envemp}{\tp{\boldsymbol{\emptyset}}}
\newcommand{\envcon}[2]{\tp{#1,}\ #2}
\newcommand{\env}{\tp{\Gamma}}
\newcommand{\typing}[2]{\tm{#1:\ }\tp{#2}}

\newcommand{\typenvcon}[2]{\tp{\Gamma,}\ \typing{#1}{#2}}
\newcommand{\sbs}[3]{#1[#2:=#3]}
\newcommand{\fv}[1]{\txt{FV}\,(#1)}
\newcommand{\fresh}[1]{\txt{Fresh}\,(#1)}
\newcommand{\txt}[1]{\text{\textit{#1}}}
\newcommand{\rewrite}[3]{#1 \mapsto_{#2} #3}
\newcommand{\reduce}[3]{#1 \rightarrow_{#2} #3}
\newcommand{\reducestar}[3]{#1 \twoheadrightarrow_{#2} #3}
\newcommand{\valuep}[1]{\txt{Value}\,(#1)}
\newcommand{\cnd}[1]{\begin{array}[t]{@{}l@{}}\txt{if}\ #1\end{array}}
\newcommand{\subformulas}[1]{\txt{Subformulas}\,(#1)}
\newcommand{\psubformulas}[1]{\txt{ProperSubformulas}\,(#1)}
\newcommand{\subterm}[1]{\txt{Subterm}\,(#1)}
\newcommand{\norm}[1]{\txt{Norm}\,(#1)}

\newcommand{\nv}{P}
               %% {\txt{N\kern-3ptV}}
               %% {\txt{N\kern-3ptW}}
\newcommand{\na}{L^{\text{\textcrlambda}}}

\subsection{The Source Language}
Metavariables $x$, $y$, and $z$ range over variables. Wherever
variables are used, renaming of bound variables is allowed;
we work up to renamings.

Types are \textbf{closed} terms defined by the abstract syntax in
Figure~\ref{fig:typ}.

\begin{figure*}[h]
\[\uncolored
\begin{array}{llll}
A,B,C & \mathbin{\ ::=\ }
         & \typone       & \textrm{unit}         \\
& \ |\ \ & \typarr{A}{B} & \textrm{function}  \\
& \ |\ \ & \typprd{A}{B} & \textrm{product}      \\
& \ |\ \ & \typsum{A}{B} & \textrm{sum}
\end{array}
\]
\caption{Types}
\label{fig:typ}
\end{figure*}

Metavariables $A$, $B$, and $C$ range over types. Types
correspond to formulas in minimal propositional logic.

Expressions are \textbf{open} terms defined by the abstract syntax in
Figure~\ref{fig:exp}.

\begin{figure*}[h]
\[\uncolored
\begin{array}{llll}
L,M,N & \mathbin{\ ::=\ }
         & \expvar{x}             & \textrm{variable}          \\
& \ |\ \ & \expunt                & \textrm{unit}              \\
& \ |\ \ & \expabs{x}{A}{N}       & \textrm{abstraction}       \\
& \ |\ \ & \expapp{L}{M}          & \textrm{application}       \\
& \ |\ \ & \expshr{x}{M}{N}       & \textrm{sharing}           \\
& \ |\ \ & \expprd{M}{N}          & \textrm{product}           \\
& \ |\ \ & \expfst{L}             & \textrm{projection-first}  \\
& \ |\ \ & \expsnd{L}             & \textrm{projection-second} \\
& \ |\ \ & \explft{M}{\tm{B}}     & \textrm{injection-left}    \\
& \ |\ \ & \exprgt{\tm{A}}{N}     & \textrm{injection-right}   \\
& \ |\ \ & \expcas{L}{x}{M}{y}{N} & \textrm{case}
\end{array}
\]
\caption{Expressions}
\label{fig:exp}
\end{figure*}

Metavariables $L$, $M$, and $N$ range over expressions. Expressions
correspond to proofs in minimal propositional logic.

%Erasing the types, colored in red, yields the usual untyped
%$\lambda$-calculus.

Capture-free substitution is defined as usual and it is denoted as
$\sbs{N}{x}{M}$ for substitution of the expression $M$ for the variable
$x$ in the expression $N$. Set of free variables in an expression is
calculated as usual and it is denote as $\fv{N}$ for set of free
variables in $N$. Moreover, the notation $count(x,N)$ is used to denote a
metafunction returning the number of times the free variable $x$ is
used syntactically in the expression $N$.
% We follow Barendregt convension such that we assume variables in a
% term are all distinct.
Type %, normalisation, and evaluation
 environments are all maps from
variable names. The metavariables $\Gamma$, and $\Delta$ range over
type environments.  The inference rules are equivalent up to exchange,
contraction, and weakening operations.

The typing judgement is defined by the inference rules in
Figure~\ref{fig:typing}.

\begin{figure*}[h]
\[\colored
\begin{array}{@{}ll@{}}
\fbox{$\env \vdash \typing{M}{A}$}
\\~\\
\inference
{\typing{x}{A} \in \env}
{\mathbf{Ax}}
{
  \env \vdash \typing{x}{A}
}
&
\inference
{}
{\typone}
{
   \env \vdash \typing{\expunt}{\typone}
}
\\~\\
\inference
{
  \typenvcon{x}{A} \vdash \typing{N}{B}
}
{{\to}\intro}
{
  \env \vdash \typing{\expabs{x}{A}{N}}{\typarr{A}{B}}
}
&
\inference
{
  \env \vdash \typing{L}{\typarr{A}{B}}
& \env \vdash \typing{M}{A}
}
{{\to}\elim}
{
  \env \vdash \typing{\expapp{L}{M}}{B}
}
\\~\\
\inference
{
  \env \vdash \typing{M}{A}
  &
  \typenvcon{x}{A} \vdash \typing{N}{B}
}
{\mathbf{let}}
{
  \env \vdash \typing{\expshr{x}{M}{N}}{B}
}
&
\inference
{
  \env \vdash \typing{M}{A}
  &
  \env \vdash \typing{N}{B}
}
{{\times}\intro}
{
  \env \vdash \typing{\expprd{M}{N}}{\typprd{A}{B}}
}
\\~\\
\inference
{
  \env \vdash \typing{L}{\typprd{A}{B}}
}
{{\times}\elim_1}
{
  \env \vdash \typing{\expfst{L}}{A}
}
&
\inference
{
  \env \vdash \typing{L}{\typprd{A}{B}}
}
{{\times}\elim_2}
{
  \env \vdash \typing{\expsnd{L}}{B}
}
\\~\\
\inference
{
  \env \vdash \typing{M}{A}
}
{{+}\intro_1}
{
  \env \vdash \typing{\explft{M}{B}}{\typsum{A}{B}}
}
&
\inference
{
  \env \vdash \typing{N}{B}
}
{{+}\intro_2}
{
  \env \vdash \typing{\exprgt{A}{N}}{\typsum{A}{B}}
}
\\~\\
\inference
{
  \env \vdash \typing{L}{\typsum{A}{B}}
&
  \typenvcon{x}{A} \vdash \typing{M}{C}
&
  \typenvcon{y}{B} \vdash \typing{N}{C}
}
{{+}\elim}
{
  \env \vdash \typing{\expcas{L}{x}{M}{y}{N}}{C}
}
\end{array}
\]
\caption{Typing Rules}
\label{fig:typing}
\end{figure*}

Typing rules are syntax-directed and algorithmic.
The correspondence to minimal logic is made evident by using colored
fonts in typing rules; removing the red bits (i.e. expressions) from
the typing rules yields the corresponding logical inference rules.\\

Values are expressions that do not compute and they follow the abstract syntax in Figure~\ref{fig:val}.

\begin{figure*}[h]
\[\uncolored
\begin{array}[t]{@{}lll@{}}
V,W & \mathbin{\ ::=\ } & \expvar{x}\ |\ \expabs{x}{A}{N}\ |\ \expunt\ |\ \expprd{V}{W}\ |\ \explft{V}{B}\ |\ \exprgt{A}{W}
\end{array}
\]
\caption{Values}
\label{fig:val}
\end{figure*}

The metavariables $V$, and $W$ range over values. The metavariable
$\nv$ range over expressions that are not values, also referred to as non-values.

\subsection{Normal Forms and Restrictions}
The normal forms that we are interested about, denoted as $N$, is defined in Figure~\ref{fig:nf}.

%% \begin{figure*}[h]
%% \[\uncolored
%% \begin{array}[t]{@{}llll@{}l@{}l@{}l@{}l@{}l@{}l@{}l@{}l@{}}
%% \ \\[-22pt]
%% (\txt{Non-Value Neutral Forms}) & Q & \mathbin{\ ::=\ } & \expapp{x}{U} & \ |\ \  & \expapp{Q}{U} & \ |\ \  & \expfst{z} & \ |\ \  &  \multicolumn{3}{@{}l@{}}{\expsnd{z}}
%% \\
%% (\txt{Value Normal Forms}) & T,U & \mathbin{\ ::=\ } & S & \ |\ \  & \multicolumn{7}{@{}l@{}}{\expabs{x}{A}{P}}
%% \\
%% (\txt{Non-Abstraction Value Normal Forms}) & R,S & \mathbin{\ ::=\ } & \expvar{x} & \ |\ \  & \expunt & \ |\ \ & \expprd{T}{U} & \ |\ \ & \explft{T}{B} & \ |\ \ & \exprgt{A}{U}
%% \\
%% (\txt{Non-Abstraction Normal Forms}) & J,K & \mathbin{\ ::=\ } & Q &\ |\ \ & S &\ |\ \ & \multicolumn{5}{@{}l@{}}{\expcas{z}{x}{J}{y}{K}}
%% \\
%%  & & \ \ \ \ | & \multicolumn{9}{l@{}}{\expshr{x}{Q}{K}}
%% \\
%% (\txt{Normal Forms}) & P & \mathbin{\ ::=\ } & K &\ |\ \ & \multicolumn{7}{@{}l@{}}{\expabs{x}{A}{P}}
%% \\
%% \end{array}
%% \]
%% \caption{Normal Forms}
%% \label{fig:nf}
%% \end{figure*}


\begin{figure*}[h]
\[\uncolored
\begin{array}[t]{@{}llll@{}l@{}l@{}l@{}l@{}l@{}l@{}l@{}l@{}l@{}l@{}}
\ \\[-22pt]
(\txt{Neutral Forms}) & Q & \mathbin{\ ::=\ } & \expapp{x}{W} & \ |\ \  & \expapp{Q}{W} & \ |\ \  & \expfst{x} & \ |\ \  &  \multicolumn{3}{@{}l@{}}{\expsnd{x}}
\\
(\txt{Value Normal Forms}) & V,W & \mathbin{\ ::=\ } & \expvar{x} & \ |\ \  & \expabs{x}{A}{N} & \ |\ \  & \expunt{} & \ |\ \  &  \expprd{V}{W}  & \ |\ \  & \explft{V}{B} & \ |\ \  & \exprgt{A}{W}
\\
(\txt{Normal Forms}) & N & \mathbin{\ ::=\ } & Q & \ |\ \  & V & \ |\ \  & \multicolumn{7}{@{}l@{}}{\expcas{z}{x}{N}{y}{M}}
\\
 & & \ \ \ \ | & \multicolumn{11}{l@{}}{\expshr{x}{Q}{N}}
\\
\end{array}
\]
\caption{Normal Forms}
\label{fig:nf}
\end{figure*}

Restricting the types of free variables, and the type of terms results
in interesting normal forms.

For a normal form which is easily compilable to \emph{C}, the
restriction $\Phi$ is applied. The restriction $\Phi$ is defined in
Figure~\ref{fig:phi}.

\begin{figure*}[h]
\[\uncolored
\begin{array}[t]{@{}llll@{}}

\inference
{}
{\Phi_0\typone}
{\Phi_0(\typone)}

&

&
\inference
{\Phi_0(A) & \Phi_0(B)}
{\Phi_0\times}
{\Phi_0(\typprd{A}{B})}
&
\inference
{\Phi_0(A) & \Phi_0(B)}
{\Phi_0+}
{\Phi_0(\typsum{A}{B})}


\\~\\

\inference
{}
{\Phi_{n+1}\typone}
{\Phi_{n+1}(\typone)}
&
\inference
{\Phi_n(A) & \Phi_{n+1}(B)}
{\Phi_{n+1}\rightarrow}
{\Phi_{n+1}(\typarr{A}{B})}
&
\inference
{\Phi_0(A) & \Phi_0(B)}
{\Phi_{n+1}\times}
{\Phi_{n+1}(\typprd{A}{B})}
&
\inference
{\Phi_0(A) & \Phi_0(B)}
{\Phi_{n+1}+}
{\Phi_{n+1}(\typsum{A}{B})}


\end{array}
\]
\caption{$\Phi$ Restriction}
\label{fig:phi}
\end{figure*}

Having $\Gamma \vdash M : B$, by restricting the type of every free
variable $\typing{x_i}{A_i} \in \Gamma$ such that $\Phi_2(A_i)$, and
by restricting the type of the term $M$ such that $\Phi_1(B)$, we have
the normal form defined in Figure~\ref{fig:rnf}. The terms of this
normal form are easily compilable to \emph{C}.  The notation
$\overline{M}$ is used to denote a non-empty sequence of terms $M_i$.

\begin{figure*}[h]
\[\uncolored
\begin{array}[t]{@{}llll@{}l@{}l@{}l@{}l@{}l@{}l@{}l@{}l@{}}
\ \\[-22pt]
(\txt{Non-Value Neutral Forms}) & Q & \mathbin{\ ::=\ } & \expapp{x}{\highlight{\overline{U}}} & \ |\ \ & \expfst{z} & \ |\ \  &  \multicolumn{3}{@{}l@{}}{\expsnd{z}}
\\
(\txt{Value Normal Forms}) & T,U & \mathbin{\ ::=\ } & S & \ |\ \  & \multicolumn{7}{@{}l@{}}{\expabs{x}{A}{P}}
\\
(\txt{Non-Abstraction Value Normal Forms}) & R,S & \mathbin{\ ::=\ } & \expvar{x} & \ |\ \  & \expunt & \ |\ \ & \expprd{\highlight{R}}{\highlight{S}} & \ |\ \ & \explft{\highlight{R}}{B} & \ |\ \ & \exprgt{A}{\highlight{S}}
\\
(\txt{Non-Abstraction Normal Forms}) & J,K & \mathbin{\ ::=\ } & Q &\ |\ \ & S &\ |\ \ & \multicolumn{5}{@{}l@{}}{\expcas{z}{x}{J}{y}{K}}
\\
 & & \ \ \ \ | & \multicolumn{9}{l@{}}{\expshr{x}{Q}{K,\ \cnd{x \in \fv{K}}}}
\\
(\txt{Normal Forms}) & P & \mathbin{\ ::=\ } & K &\ |\ \ & \multicolumn{7}{@{}l@{}}{\expabs{x}{A}{P}}
\\
\end{array}
\]
\caption{$\Phi-$Restricted Normal Forms}
\label{fig:rnf}
\end{figure*}

\subsection{Normalisation}
Rewrite rules, denoted as $\rewrite{}{i}{}$ for phase $i$, are defined in
Figure~\ref{fig:red}. The metavariables $F$, and $G$ denote
capture-allowing evaluation frames. The metafunctions defined for
expressions carry over to evaluation frames trivially.

\begin{figure*}[h]
\[\uncolored
\begin{array}[t]{@{}lllll@{}}
\multicolumn{5}{@{}c@{}}{\txt{Phase 1}}\\[-9pt]
\multicolumn{5}{@{}c@{}}{\rule{120pt}{1pt}}\\[1pt]
(\eta_{\rightarrow})
& L
& \ \rewrite{}{1}{}\ \
& \expabs{x}{A}{\expapp{L}{x}}
& %\cnd{%% \neg\valuep{M},\ and\\\ \ \ \fresh{x}}
\\[0pt]
\multicolumn{5}{@{}l@{}}{\txt{\ \ \ \ where\ \ \ }\env \vdash \typing{L}{\typarr{A}{B}},\ L \neq \expabs{y}{A}{N}\txt{, and }\fresh{x}}
\\[10pt]

\multicolumn{5}{@{}c@{}}{\txt{Phase 2}}\\[-9pt]
\multicolumn{5}{@{}c@{}}{\rule{120pt}{1pt}}\\[1pt]

\multicolumn{5}{@{}l@{}}{F \mathbin{\ ::=\ }
    \expapp{M}{\hole}\
 |\ \expprd{\hole}{N}\
 |\ \expprd{V}{\hole}\
 |\ \expfst{\hole}\
 |\ \expsnd{\hole}\
 |\ \explft{\hole}{B}\
 |\ \exprgt{A}{\hole}\
 |\ \expcasind{\hole}{x}{M}{y}{N}}\\[1pt]

(\txt{let}_{\rightarrow\times+})
& F[\nv]
& \ \rewrite{}{2}{}\ \
& \expshr{x}{\nv}{F[x]}
& \cnd{%% \neg\valuep{M},\ and\\\ \ \
       \fresh{x}} \\[8pt]

\multicolumn{5}{@{}c@{}}{\txt{Phase 3}}\\[-9pt]
\multicolumn{5}{@{}c@{}}{\rule{120pt}{1pt}}\\[1pt]

\multicolumn{5}{@{}l@{}}{G \mathbin{\ ::=\ }
    \expapp{\hole}{V}\
 |\ \expshr{x}{\hole}{N}}\\[1pt]

(\kappa_{\txt{let}})
& G[\expshr{x}{\nv}{N}]
& \ \rewrite{}{3}{}\ \
& \expshr{x}{\nv}{G[N]}
& \cnd{x \notin \fv{G}}\\[0pt]

(\kappa_{\txt{case}})
& G[\expcasind{z}{x}{M}{y}{N}]
& \ \rewrite{}{3}{}\ \
& \expcasind{z}{x}{G[M]}{y}{G[N]}
& \cnd{x \notin \fv{G},\ \txt{and}\\\ \ \ \, y \notin \fv{G}} \\[20pt]

%% \multicolumn{5}{@{}l@{}}{H \mathbin{\ ::=\ } \expshr{x}{M}{\hole}\ |\ \expcas{z}{x}{\hole}{y}{N}\ |\ \expcas{z}{x}{M}{y}{\hole}}\\[1pt]

%% (\kappa_{\lambda})
%% & H[\expabs{x}{A}{N}]
%% & \ \rewrite{}{}{}\ \
%% & \expabs{x}{A}{H[N]}
%% & \cnd{x \notin \fv{H}} \\[8pt]


(\beta_{\rightarrow})
& \expapp{(\expabs{x}{A}{N})}{V}
& \ \rewrite{}{3}{}\ \
& \sbs{N}{x}{V}
& \\[0pt]

(\beta_{\times_1})
& \expfst{\expprd{V}{W}}
& \ \rewrite{}{3}{}\ \
& V
& \\[0pt]

(\beta_{\times_2})
& \expsnd{\expprd{V}{W}}
& \ \rewrite{}{3}{}\ \
& W
& \\[0pt]

(\beta_{+_1})
& \expcasind{(\explft{V}{B})}{x}{M}{y}{N}
& \ \rewrite{}{3}{}\ \
& \sbs{M}{x}{V}
& \\[0pt]

(\beta_{+_2})
& \expcasind{(\exprgt{A}{W})}{x}{M}{y}{N}
& \ \rewrite{}{3}{}\ \
& \sbs{N}{y}{W}
& \\[0pt]

(\beta_{\txt{let}})
& \expshr{x}{V}{N}
& \ \rewrite{}{3}{}\ \
& \sbs{N}{x}{V}
& \\[8pt]

\multicolumn{5}{@{}c@{}}{\txt{Phase 4}}\\[-9pt]
\multicolumn{5}{@{}c@{}}{\rule{120pt}{1pt}}\\[1pt]

(\txt{need})
& \expshr{x}{\nv}{N}
& \ \rewrite{}{4}{}\ \
& N %%\sbs{N}{x}{\nv}
& \cnd{%% \neg\valuep{M},\ and\\\ \ \
       %%Count(x,N) < 2
       x \notin \fv{N}}\\[0pt]
\end{array}
\]
\caption{Rewrite Rules}
\label{fig:red}
\end{figure*}

One-step reduction relation, denoted as $\reduce{}{i}{}$ for phase
$i$, is compatible closure of the corresponding rewrite rule
$\rewrite{}{i}{}$. One-step reduction relation for phase 1 has an
extra condition such that a redex should not be immediately at the
left-hand side of an application, i.e. $E \neq
{E}'[\expapp{\hole}{M}]$ for the compatible contexts $E$ and $E'$.

Reduction relation, denoted as $\reducestar{}{i}{}$ for phase $i$, is
reflexive transitive closure of the corresponding one-step reduction
relation $\reduce{}{i}{}$. The overall normalisation, denoted as
$\reducestar{}{}{}$, is composition of the reduction for all four
phases,
i.e. $\reducestar{}{4}{}\circ\reducestar{}{3}{}\circ\reducestar{}{2}{}\circ\reducestar{}{1}{}$.

\subsection{Theorems and Proofs}

\begin{proposition}[Normal Form Syntax]\ \\
\label{prop_normal}
An expression $M$ is in normal form, if and only if, it matches the
abstract syntax in Figure~\ref{fig:nf}, given by the entry $N$.
\end{proposition}


\begin{proposition}[Preservation]\ \\
\label{prop_preservation}
Reduction preserves the type of the source expression:

 if $\env \vdash \typing{M}{A}$ and $\reducestar{M}{}{N}$, then
$\env \vdash \typing{N}{A}$.
\end{proposition}
