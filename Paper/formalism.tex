%\newcommand{\incolor}[1]{#1}    % Use to typeset in color
\newcommand{\incolor}[1]{}     % Use to typeset in black and white

% color framework

% boldsymbol is evil!
\renewcommand{\boldsymbol}[1]{#1}

\newcommand{\judgecolor}{}
\newcommand{\typecolor}{}
\newcommand{\termcolor}{}
\newcommand{\Typecolor}{}
\newcommand{\Termcolor}{}

\newcommand{\uncolored}{
  \incolor{
    \renewcommand{\judgecolor}{}
    \renewcommand{\typecolor}{}
    \renewcommand{\termcolor}{}
    \renewcommand{\Typecolor}{}
    \renewcommand{\Termcolor}{}
  }
}

\newcommand{\colored}{
  \incolor{
    \renewcommand{\judgecolor}{\color{black}}
    \renewcommand{\typecolor}{\color{blue}}
    \renewcommand{\termcolor}{\color{red}}
    \renewcommand{\Typecolor}{\color{cyan}}
    \renewcommand{\Termcolor}{\color{magenta}}
  }
}

\newcommand{\tp}[1]{{\typecolor #1}}
\newcommand{\tm}[1]{{\termcolor #1}}

\newcommand{\reducedstrut}{\vrule width 0pt height .9\ht\strutbox depth .9\dp\strutbox\relax}
\definecolor{lightgray}{gray}{0.7}
\newcommand{\highlight}[1]{%
  \begingroup
  \setlength{\fboxsep}{0pt}%
  \colorbox{lightgray}{\reducedstrut\ensuremath{#1}\/}%
  \endgroup
}

\newcommand{\etal}{\emph{et~al.}}
\newcommand{\inference}[3]{\infer[\mathsf{#2}]{#3}{#1}}
\newcommand{\Inference}[3]{\infer=[\mathsf{#2}]{#3}{#1}}
\newcommand{\hole}{[\ ]}
\newcommand{\intro}{\mathcal{I}}
\newcommand{\elim}{\mathcal{E}}
\newcommand{\nv}{P}

%% Types
\newcommand{\typvar}[1]{#1}
\newcommand{\typzro}{\mathbf{0}}
\newcommand{\typone}{\mathbf{1}}
\newcommand{\typarr}[2]{#1\boldsymbol{\rightarrow}#2}
\newcommand{\typprd}[2]{#1\boldsymbol{\times}#2}
\newcommand{\typsum}[2]{#1\boldsymbol{+}#2}
\newcommand{\typrec}[2]{\boldsymbol{\mu}#1\boldsymbol{.}#2}

%% Expressions
\newcommand{\expvar}[1]{#1}
\newcommand{\expunt}{\boldsymbol{()}}
\newcommand{\expabs}[3]{\boldsymbol{\lambda}#1\boldsymbol{.}#3}
\newcommand{\expapp}[2]{#1\ #2}
\newcommand{\expshr}[3]{\mathbf{let}\ #1\boldsymbol{=}#2\ \mathbf{in}\ #3}
\newcommand{\expshrind}[3]{\begin{array}[t]{@{}l@{}}\mathbf{let}\ #1\boldsymbol{=}#2\\ \mathbf{in}\ \ #3\end{array}}
\newcommand{\expprd}[2]{\boldsymbol{(}#1 \boldsymbol{,} #2\boldsymbol{)}}
\newcommand{\expfst}[1]{\mathbf{fst}\ #1}
\newcommand{\expsnd}[1]{\mathbf{snd}\ #1}
\newcommand{\explft}[2]{\mathbf{inl}\ #1}
\newcommand{\exprgt}[2]{\mathbf{inr}\ #2}
\newcommand{\expcas}[5]{\mathbf{case}\ #1\ \mathbf{of}\ \boldsymbol{\{}\mathbf{inl}\ #2.#3 \boldsymbol{;} \mathbf{inr}\ #4.#5\boldsymbol{\}}}

%% Environment
\newcommand{\envemp}{\tp{\boldsymbol{\emptyset}}}
\newcommand{\envcon}[2]{\tp{#1,}\ #2}
\newcommand{\env}{\tp{\Gamma}}
\newcommand{\typing}[2]{\tm{#1:\ }\tp{#2}}

\newcommand{\typenvcon}[2]{\tp{\Gamma,}\ \typing{#1}{#2}}
\newcommand{\sbs}[3]{#1[#2:=#3]}
\newcommand{\fv}[1]{\txt{FV}(#1)}
\newcommand{\fresh}[1]{#1 \txt{ fresh}}
\newcommand{\txt}[1]{\text{\textit{#1}}}
\newcommand{\rewrite}[3]{#1 \mapsto_{#2} #3}
\newcommand{\reduce}[3]{#1 \rightarrow_{#2} #3}
\newcommand{\reducestar}[3]{#1 \twoheadrightarrow_{#2} #3}
\newcommand{\valuep}[1]{\txt{Value}\,(#1)}
\newcommand{\cnd}[1]{#1}
%\newcommand{\cnd}[1]{\begin{array}[t]{@{}l@{}}\txt{if}\ #1\end{array}}
\newcommand{\subformulas}[1]{\txt{Subformulas}\,(#1)}
\newcommand{\psubformulas}[1]{\txt{ProperSubformulas}\,(#1)}
\newcommand{\subterm}[1]{\txt{Subterm}\,(#1)}
\newcommand{\norm}[1]{\txt{Norm}\,(#1)}

\newcommand{\na}{L^{\text{\textcrlambda}}}


\newcommand{\figterm}{
\begin{figure*}[t]
\[\uncolored
\begin{array}{l@{\quad}rcl}
\text{Types} & A,B,C & ::=&
% \typone             %   & \textrm{unit}      \\
  \iota         \mid  %   & \textrm{base type} \\
  \typarr{A}{B} \mid  %   & \textrm{function}  \\
  \typprd{A}{B} \mid  %   & \textrm{product}   \\
  \typsum{A}{B}       %   & \textrm{sum}
\\[1ex]
\text{Terms} & L,M,N & ::= &
% \expunt                \mid   % \textrm{unit}              \\
  \expvar{x}             \mid   % \textrm{variable}          \\
  c~\overline{M}         \mid   % constants
  \expabs{x}{A}{N}       \mid   % \textrm{abstraction}       \\
  \expapp{L}{M}          \mid   % \textrm{application}       \\
  \expshr{x}{M}{N}       \mid   % \textrm{sharing}           \\
  \expprd{M}{N}          \mid   % \textrm{product}           \\
  \expfst{L}             \mid   % \textrm{projection-first}  \\
  \expsnd{L}             \\&&&  % \textrm{projection-second} \\
  \explft{M}{\tm{B}}     \mid   % \textrm{injection-left}    \\
  \exprgt{\tm{A}}{N}     \mid   % \textrm{injection-right}   \\
  \expcas{L}{x}{M}{y}{N}        % \textrm{case}
\\[1ex]
\text{Values} & V,W & ::= &
% \expunt\               \mid
  \expvar{x}             \mid
  \expabs{x}{A}{N}       \mid
  \expprd{V}{W}          \mid
  \explft{V}{B}          \mid
  \exprgt{A}{W}
\end{array}
\]
\caption{Types, Terms, and Values}
\label{fig:term}
\end{figure*}
}

\newcommand{\figtyping}{
\begin{figure*}[h]
\[\colored
\begin{array}{@{}ll@{}}
\fbox{$\env \vdash \typing{M}{A}$}
\\~\\
\inference
{\typing{x}{A} \in \env}
{\mathbf{Ax}}
{
  \env \vdash \typing{x}{A}
}
&
\inference
{}
{\typone}
{
   \env \vdash \typing{\expunt}{\typone}
}
\\~\\
\inference
{
  \typenvcon{x}{A} \vdash \typing{N}{B}
}
{{\to}\intro}
{
  \env \vdash \typing{\expabs{x}{A}{N}}{\typarr{A}{B}}
}
&
\inference
{
  \env \vdash \typing{L}{\typarr{A}{B}}
& \env \vdash \typing{M}{A}
}
{{\to}\elim}
{
  \env \vdash \typing{\expapp{L}{M}}{B}
}
\\~\\
\inference
{
  \env \vdash \typing{M}{A}
  &
  \typenvcon{x}{A} \vdash \typing{N}{B}
}
{\mathbf{let}}
{
  \env \vdash \typing{\expshr{x}{M}{N}}{B}
}
&
\inference
{
  \env \vdash \typing{M}{A}
  &
  \env \vdash \typing{N}{B}
}
{{\times}\intro}
{
  \env \vdash \typing{\expprd{M}{N}}{\typprd{A}{B}}
}
\\~\\
\inference
{
  \env \vdash \typing{L}{\typprd{A}{B}}
}
{{\times}\elim_1}
{
  \env \vdash \typing{\expfst{L}}{A}
}
&
\inference
{
  \env \vdash \typing{L}{\typprd{A}{B}}
}
{{\times}\elim_2}
{
  \env \vdash \typing{\expsnd{L}}{B}
}
\\~\\
\inference
{
  \env \vdash \typing{M}{A}
}
{{+}\intro_1}
{
  \env \vdash \typing{\explft{M}{B}}{\typsum{A}{B}}
}
&
\inference
{
  \env \vdash \typing{N}{B}
}
{{+}\intro_2}
{
  \env \vdash \typing{\exprgt{A}{N}}{\typsum{A}{B}}
}
\\~\\
\inference
{
  \env \vdash \typing{L}{\typsum{A}{B}}
&
  \typenvcon{x}{A} \vdash \typing{M}{C}
&
  \typenvcon{y}{B} \vdash \typing{N}{C}
}
{{+}\elim}
{
  \env \vdash \typing{\expcas{L}{x}{M}{y}{N}}{C}
}
\end{array}
\]
\caption{Typing Rules}
\label{fig:typing}
\end{figure*}
}

\newcommand{\fignf}{
\begin{figure*}[t]
\[\uncolored
\begin{array}{l@{\quad}rcl}
\text{Neutral Forms}
  & Q & ::= & \expapp{x}{W}
         \mid \expapp{c}{\overline{W}}
         \mid \expapp{Q}{W}
         \mid \expfst{x}
         \mid \expsnd{x}      
\\[1ex]
\text{Normal Values}
  & V,W & ::= & \expvar{x}
           \mid \expabs{x}{A}{N}
           \mid \expprd{V}{W}
           \mid \explft{V}{B} 
           \mid \exprgt{A}{W}
\\[1ex]
\text{Normal Forms} & N,M & ::= &
   Q  \mid  V \mid \expcas{z}{x}{N}{y}{M} \mid \expshr{x}{Q}{N}
\\
\end{array}
\]
\caption{Normal Forms}
\label{fig:nf}
\end{figure*}
}

\newcommand{\fignorm}{
\begin{figure*}[t]
%% \multicolumn{5}{@{}c@{}}{\txt{Phase 1}}\\[-9pt]
%% \multicolumn{5}{@{}c@{}}{\rule{120pt}{1pt}}\\[1pt]
%% (\eta_{\rightarrow})
%% & L
%% & \ \rewrite{}{1}{}\ \
%% & \expabs{x}{A}{\expapp{L}{x}}
%% & %\cnd{%% \neg\valuep{M},\ and\\\ \ \ \fresh{x}}
%% \\[0pt]
%% \multicolumn{5}{@{}l@{}}{\txt{\ \ \ \ where\ \ \ }\env \vdash \typing{L}{\typarr{A}{B}},\ L \neq \expabs{y}{A}{N}\txt{, and }\fresh{x}}
%% \\[10pt]

Phase 1 (let-insertion)
\[
  \begin{array}{lcl}
   F &\mathbin{::=}&
       \expapp{c}{(\overline{M},\hole,\overline{N})}
  \mid \expapp{M}{\hole}
  \mid \expprd{\hole}{N}
  \mid \expprd{M}{\hole}
  \mid \expfst{\hole}
  \mid \expsnd{\hole} \\
 &\mid&\explft{\hole}{B}
  \mid \exprgt{A}{\hole}
  \mid \expcas{\hole}{x}{M}{y}{N} \\
 \end{array}
\]%
\[
\begin{array}[t]{@{}lll@{\quad}l@{}}
(\txt{let})
& F[\nv]
& \ \rewrite{}{1}{}\ \
& \expshr{x}{\nv}{F[x]}, \quad x \text{ fresh} \\
\end{array}
\]

\vspace{2ex}

Phase 2 (symbolic evaluation)
\[
G \mathbin{::=}
    \expapp{\hole}{V} \mid \expshr{x}{\hole}{N}
\]
\[
\begin{array}[t]{@{\quad}llll@{}}
(\kappa.{\txt{let}})
& G[\expshr{x}{\nv}{N}]
& \ \rewrite{}{2}{}\ \
& \expshr{x}{\nv}{G[N]}, \\
&&&\quad x \notin \fv{G}\\

(\kappa.{\txt{case}})
& G[\expcas{z}{x}{M}{y}{N}] & \ \rewrite{}2{} \\
\multicolumn{3}{@{}l@{}}
{\qquad\qquad\qquad\quad \expcas{z}{x}{G[M]}{y}{G[N]},} & \quad x,y \notin \fv{G} \\[1ex]

%% \multicolumn{5}{@{}l@{}}{H \mathbin{\ ::=\ } \expshr{x}{M}{\hole}\ |\ \expcas{z}{x}{\hole}{y}{N}\ |\ \expcas{z}{x}{M}{y}{\hole}}\\[1pt]

%% (\kappa_{\lambda})
%% & H[\expabs{x}{A}{N}]
%% & \ \rewrite{}{}{}\ \
%% & \expabs{x}{A}{H[N]}
%% & \cnd{x \notin \fv{H}} \\[8pt]


(\beta.{\rightarrow})
& \expapp{(\expabs{x}{A}{N})}{V}
& \ \rewrite{}{2}{}\ \
& \sbs{N}{x}{V} \\

(\beta.{\times_1})
& \expfst{\expprd{V}{W}}
& \ \rewrite{}{2}{}\ \
& V \\

(\beta.{\times_2})
& \expsnd{\expprd{V}{W}}
& \ \rewrite{}{2}{}\ \
& W \\

(\beta.{+_1})
& \expcas{(\explft{V}{B})}{x}{M}{y}{N}
& \ \rewrite{}{2}{}\ \
& \sbs{M}{x}{V} \\

(\beta.{+_2})
& \expcas{(\exprgt{A}{W})}{x}{M}{y}{N}
& \ \rewrite{}{2}{}\ \
& \sbs{N}{y}{W} \\

(\beta.{\txt{let}})
& \expshr{x}{V}{N}
& \ \rewrite{}{2}{}\ \
& \sbs{N}{x}{V} \\
\end{array}
\]

\vspace{2ex}

Phase 3 (garbage collection)
\[
\begin{array}[t]{@{}llll@{\quad}l@{}}

(\txt{need})
& \expshr{x}{\nv}{N}
& \ \rewrite{}{3}{}\ \
& N %%\sbs{N}{x}{\nv}
& \cnd{%% \neg\valuep{M},\ and\\\ \ \
       %%Count(x,N) < 2
       x \notin \fv{N}}\\
\end{array}
\]
\caption{Normalisation rules}
\label{fig:norm}
\end{figure*}
}

%%%%%%%%%%%%%%%%%%%%%%%%%%%%%%%%%%%%%%%%%%%%%%%%%%%%%%%%%%%%%%%%%%%%%%%%

%\todo{Removed unit type. Either remove unit type or add empty type.}

This section introduces a collection of reduction rules for
normalising terms that enforces the subformula property 
while ensuring sharing is preserved. The rules adapt to
both call-by-need and call-by-value.

We work with simple types. The only polymorphism in our examples
corresponds to instantiating constants (such as $\mathit{while}$) at
different types.

Types, terms, and values are presented in Figure~\ref{fig:term}. We
let $A$, $B$, $C$ range over types, including base types ($\iota$),
functions ($A \to B$), products ($A \times B$), and sums ($A + B$).
We let $L$, $M$, $N$ range over terms, and $x$, $y$, $z$ range over
variables.  We let $c$ range over primitive constants, which are fully
applied (applied to a sequence of terms of length equal to the
constant's arity).  We follow the usual convention that terms are
equivalent up to renaming of bound variables. We write $\fv{N}$ for
the set of free variables of $N$, and $\sbs{N}{x}{M}$ for
capture-avoiding substitution of $M$ for $x$ in $N$.
%
%% [SL: not relevant]
%% Types correspond to propositions and terms to proofs in
%% minimal propositional logic.
%
We let $V$, $W$ range over values, and $\nv$ range over non-values
(that is, any term that is not a value).

We let $\Gamma$ range over type environments, which are sets of pairs
of variables with types $x:A$. We write $\Gamma \vdash M:A$ to
indicate that term $M$ has type $A$ under type environment
$\Gamma$. The typing rules are standard.

The grammar of normal forms is given in Figure~\ref{fig:nf}. We reuse
$L,M,N$ to range over terms in normal form and $V,W$ to range over
values in normal form, and we let $Q$ range over neutral forms.

Reduction rules for normalisation are presented in
Figure~\ref{fig:norm}, broken into three phases. We write $M
\mapsto_i N$ to indicate that $M$ reduces to $N$ in phase $i$. We let
$F$ and $G$ range over two different forms of evaluation frame used in
Phases~2 and~3 respectively. We write $\fv{F}$ for the set of free
variables of $F$, and similarly for $G$.
The reduction relation is closed under compatible closure.

\figterm
\fignf
\fignorm

The normalisation procedure consists of exhaustively applying
the reductions of Phase~1 until no more apply, then similarly for
Phase~2, and finally for Phase~3. Phase~1 performs let-insertion,
naming subterms that are not values.
%% along the lines of a translation to A-normal form
%% \citep{a-normal-form} or reductions (let.1) and (let.2) in Moggi's
%% metalanguage for monads \citep{moggi}.
Phase~2 performs standard $\beta$ and commuting reductions, and is the
only phase that is crucial obtaining normal forms that satisfy the
subformula property. Phase~3 ``garbage collects'' unused terms, as in
the call-by-need lambda calculus \citep{call-by-need}. Phase~3 may be
omitted if the intended semantics of the target language is
call-by-value rather than call-by-need.

Every term has a normal form.
\begin{proposition}[Strong normalisation]
Each of the reduction relations $\rewrite{}{i}{}$ is strongly
normalising: all $\rewrite{}{i}{}$ reduction sequences on well-typed
terms are finite.
\end{proposition}
The only non-trivial proof is for $\rewrite{}{2}{}$, which can be
proved via a standard reducibility argument
(e.g. \cite{Lindley07}). If the target language includes general
recursion, normalisation should treat the fixpoint operator as an
uninterpreted constant.

The grammar of Figure~\ref{fig:nf} characterises normal forms
precisely.
\begin{proposition}[Normal Form Syntax]
\label{prop_normal}
An expression $N$ matches the syntax of normal forms in
Figure~\ref{fig:nf} if and only if it is in normal form with regard to
the reduction rules of Figure~\ref{fig:norm}.
\end{proposition}

The \emph{subformulas} of a type are the type itself and its
components; for instance, the subformulas of $A \to B$ are $A \to B$
itself and the subformulas of $A$ and $B$. The \emph{proper
  subformulas} of a type are all its subformulas other than the type
itself.  Terms in normal form satisfy the subformula property.

\begin{proposition}[Subformula property]
\label{prop_subformula}
If $\Gamma \vdash M:A$ and the normal form of $M$ is $N$ by the
reduction rules of Figure~\ref{fig:norm}, then $\Gamma \vdash N:A$ and
every subterm of $N$ has a type that is either a subformula of $A$ or
a subformula of a type in $\Gamma$.  Further, every subterm other than
$N$ itself and free variables of $N$ has a type that is a proper
subformula of $A$ or a proper subformula of a type in $\Gamma$.
\end{proposition}




