%% Types
\newcommand{\zero}{\mathbf{0}}
\newcommand{\one}{\mathbf{1}}
\newcommand{\arrow}[2]{#1\rightarrow#2}
\newcommand{\product}[2]{#1\times#2}
\newcommand{\coproduct}[2]{#1+#2}
\newcommand{\rec}[2]{\mu#1.#2}

%% Expressions
\newcommand{\key}{\mathbf}

\newcommand{\expvar}[1]{#1}
\newcommand{\expconst}[2]{#1~#2}
\newcommand{\expunit}{()}
\newcommand{\expabs}[3]{\lambda#1.#3}
\newcommand{\expapp}[2]{#1~#2}
\newcommand{\explet}[3]{\key{let}\ #1=#2\ \key{in}\ #3}
\newcommand{\exppair}[2]{(#1, #2)}
\newcommand{\expfst}[1]{\key{fst}~#1}
\newcommand{\expsnd}[1]{\key{snd}~#1}
\newcommand{\expinl}[2]{\key{inl}~#1}
\newcommand{\expinr}[2]{\key{inr}~#2}
\newcommand{\expcase}[5]{\key{case}\ #1\ \key{of}\ \{\key{inl}\ #2.#3; \key{inr}\ #4.#5\}}

%% Meta language stuff
\newcommand{\subst}[3]{#1[#2:=#3]}
\newcommand{\fv}[1]{\mathit{FV}(#1)}
\newcommand{\rewrite}[1]{\mathbin{\mapsto_{#1}}}
\newcommand{\hole}{[~]}
\newcommand{\nv}{P}

%% Typing
\newcommand{\intro}{\mathcal{I}}
\newcommand{\elim}{\mathcal{E}}
\newcommand{\inference}[3]{\infer[\mathsf{#2}]{#3}{#1}}

\newcommand{\figterm}{
\begin{figure*}[t]
\[
\begin{array}{l@{\quad}rcl}
\text{Types} & A,B,C & ::=&
% \one
  \iota           \mid
  \arrow{A}{B}    \mid
  \product{A}{B}  \mid
  \coproduct{A}{B}
\\[1ex]
\text{Terms} & L,M,N & ::= &
% \expunit                \mid
  \expvar{x}              \mid
  \expconst{c}{M}         \mid
  \expabs{x}{A}{N}        \mid
  \expapp{L}{M}           \mid
  \explet{x}{M}{N}        \mid
  \exppair{M}{N}          \mid
  \expfst{L}              \mid
  \expsnd{L}              \\&&&
  \expinl{M}{\tm{B}}      \mid
  \expinr{\tm{A}}{N}      \mid
  \expcase{L}{x}{M}{y}{N}
\\[1ex]
\text{Values} & V,W & ::= &
% \expunit         \mid
  \expvar{x}       \mid
  \expabs{x}{A}{N} \mid
  \exppair{V}{W}   \mid
  \expinl{V}{B}    \mid
  \expinr{A}{W}
\end{array}
\]
\caption{Types, Terms, and Values}
\label{fig:term}
\end{figure*}
}

\newcommand{\figtyping}{
\begin{figure*}[h]
\[
\begin{array}{@{}ll@{}}
\fbox{$\Gamma \vdash M:A$}
\\~\\
\inference
{x:A \in \Gamma}
{\mathbf{Ax}}
{
  \Gamma \vdash x:A
}
&
\inference
{}
{\one}
{
   \Gamma \vdash \expunit:\one
}
\\~\\
\inference
{
  \Gamma, x:A \vdash N:B
}
{{\to}\intro}
{
  \Gamma \vdash \expabs{x}{A}{N}:\arrow{A}{B}
}
&
\inference
{
  \Gamma \vdash L:\arrow{A}{B}
& \Gamma \vdash M:A
}
{{\to}\elim}
{
  \Gamma \vdash \expapp{L}{M}:B
}
\\~\\
\inference
{
  \Gamma \vdash M:A
  &
  \Gamma, x:A \vdash N:B
}
{\mathbf{let}}
{
  \Gamma \vdash \explet{x}{M}{N}:B
}
&
\inference
{
  \Gamma \vdash M:A
  &
  \Gamma \vdash N:B
}
{{\times}\intro}
{
  \Gamma \vdash \exppair{M}{N}:\product{A}{B}
}
\\~\\
\inference
{
  \Gamma \vdash L:\product{A}{B}
}
{{\times}\elim_1}
{
  \Gamma \vdash \expfst{L}:A
}
&
\inference
{
  \Gamma \vdash L:\product{A}{B}
}
{{\times}\elim_2}
{
  \Gamma \vdash \expsnd{L}:B
}
\\~\\
\inference
{
  \Gamma \vdash M:A
}
{{+}\intro_1}
{
  \Gamma \vdash \expinl{M}{B}:\coproduct{A}{B}
}
&
\inference
{
  \Gamma \vdash N:B
}
{{+}\intro_2}
{
  \Gamma \vdash \expinr{A}{N}:\coproduct{A}{B}
}
\\~\\
\inference
{
  \Gamma \vdash L:\coproduct{A}{B}
&
  \Gamma, x:A \vdash M:C
&
  \Gamma, y:B \vdash N:C
}
{{+}\elim}
{
  \Gamma \vdash \expcase{L}{x}{M}{y}{N}:C
}
\end{array}
\]
\caption{Typing Rules}
\label{fig:typing}
\end{figure*}
}

\newcommand{\fignf}{
\begin{figure*}[t]
\[
\begin{array}{l@{\quad}rcl}
\text{Neutral Forms}
  & Q & ::= & \expapp{x}{W}
         \mid \expconst{c}{\overline{W}}
         \mid \expapp{Q}{W}
         \mid \expfst{x}
         \mid \expsnd{x}      
\\[1ex]
\text{Normal Values}
  & V,W & ::= & \expvar{x}
           \mid \expabs{x}{A}{N}
           \mid \exppair{V}{W}
           \mid \expinl{V}{B}
           \mid \expinr{A}{W}
\\[1ex]
\text{Normal Forms} & N,M & ::= &
   Q  \mid  V \mid \expcase{z}{x}{N}{y}{M} \mid \explet{x}{Q}{N}
\\
\end{array}
\]
\caption{Normal Forms}
\label{fig:nf}
\end{figure*}
}

\newcommand{\fignorm}{
\begin{figure*}[t]
Phase 1 (let-insertion)
\[
  \begin{array}{lcl}
   F &\mathbin{::=}&
       \expconst{c}{(\overline{M},\hole,\overline{N})}
  \mid \expapp{M}{\hole}
  \mid \exppair{\hole}{N}
  \mid \exppair{M}{\hole}
  \mid \expfst{\hole}
  \mid \expsnd{\hole} \\
 &\mid&\expinl{\hole}{B}
  \mid \expinr{A}{\hole}
  \mid \expcase{\hole}{x}{M}{y}{N} \\
 \end{array}
\]%
\[
(\mathit{let})
~~F[\nv]
~\rewrite{1}~
\explet{x}{\nv}{F[x]}, \quad x \text{ fresh}
\]

\vspace{2ex}

Phase 2 (symbolic evaluation)
\[
G \mathbin{::=}
    \expapp{\hole}{V} \mid \explet{x}{\hole}{N}
\]
\[
\begin{array}[t]{@{\quad}llll@{}}
(\kappa.{\mathit{let}})
& G[\explet{x}{\nv}{N}]
& \rewrite{2}
& \explet{x}{\nv}{G[N]}, \\
&&&\quad x \notin \fv{G}\\

(\kappa.{\mathit{case}})
& G[\expcase{z}{x}{M}{y}{N}] & \rewrite{}2{} \\
\multicolumn{3}{@{}l@{}}
{\qquad\qquad\qquad\quad \expcase{z}{x}{G[M]}{y}{G[N]},} & \quad x,y \notin \fv{G} \\[1ex]

(\beta.{\rightarrow})
& \expapp{(\expabs{x}{A}{N})}{V}
& \rewrite{2}
& \subst{N}{x}{V} \\

(\beta.{\times_1})
& \expfst{\exppair{V}{W}}
& \rewrite{2}
& V \\

(\beta.{\times_2})
& \expsnd{\exppair{V}{W}}
& \rewrite{2}
& W \\

(\beta.{+_1})
& \expcase{(\expinl{V}{B})}{x}{M}{y}{N}
& \rewrite{2}
& \subst{M}{x}{V} \\

(\beta.{+_2})
& \expcase{(\expinr{A}{W})}{x}{M}{y}{N}
& \rewrite{2}
& \subst{N}{y}{W} \\

(\beta.{\mathit{let}})
& \explet{x}{V}{N}
& \rewrite{2}
& \subst{N}{x}{V} \\
\end{array}
\]

\vspace{2ex}

Phase 3 (garbage collection)
\[
\begin{array}[t]{@{}llll@{\quad}l@{}}

(\mathit{need})
& \explet{x}{\nv}{N}
& \rewrite{3}
& N,
& x \notin \fv{N}\\
%% & \subst{N}{x}{\nv},
%% & Count(x,N) < 2 \\
\end{array}
\]
\caption{Normalisation rules}
\label{fig:norm}
\end{figure*}
}

%%%%%%%%%%%%%%%%%%%%%%%%%%%%%%%%%%%%%%%%%%%%%%%%%%%%%%%%%%%%%%%%%%%%%%%%

%\todo{Removed unit type. Either remove unit type or add empty type.}

This section introduces a collection of reduction rules for
normalising terms that enforces the subformula property 
while ensuring sharing is preserved. The rules adapt to
both call-by-need and call-by-value.

We work with simple types. The only polymorphism in our examples
corresponds to instantiating constants (such as $\mathit{while}$) at
different types.

Types, terms, and values are presented in Figure~\ref{fig:term}. We
let $A$, $B$, $C$ range over types, including base types ($\iota$),
functions ($A \to B$), products ($A \times B$), and sums ($A + B$).
We let $L$, $M$, $N$ range over terms, and $x$, $y$, $z$ range over
variables.  We let $c$ range over primitive constants, which are fully
applied (applied to a sequence of terms of length equal to the
constant's arity).  We follow the usual convention that terms are
equivalent up to renaming of bound variables. We write $\fv{N}$ for
the set of free variables of $N$, and $\subst{N}{x}{M}$ for
capture-avoiding substitution of $M$ for $x$ in $N$.
%
%% [SL: not relevant]
%% Types correspond to propositions and terms to proofs in
%% minimal propositional logic.
%
We let $V$, $W$ range over values, and $\nv$ range over non-values
(that is, any term that is not a value).

We let $\Gamma$ range over type environments, which are sets of pairs
of variables with types $x:A$. We write $\Gamma \vdash M:A$ to
indicate that term $M$ has type $A$ under type environment
$\Gamma$. The typing rules are standard.

The grammar of normal forms is given in Figure~\ref{fig:nf}. We reuse
$L,M,N$ to range over terms in normal form and $V,W$ to range over
values in normal form, and we let $Q$ range over neutral forms.

Reduction rules for normalisation are presented in
Figure~\ref{fig:norm}, broken into three phases. We write $M
\mapsto_i N$ to indicate that $M$ reduces to $N$ in phase $i$. We let
$F$ and $G$ range over two different forms of evaluation frame used in
Phases~2 and~3 respectively. We write $\fv{F}$ for the set of free
variables of $F$, and similarly for $G$.
The reduction relation is closed under compatible closure.

\figterm
\fignf
\fignorm
%\figtyping

The normalisation procedure consists of exhaustively applying
the reductions of Phase~1 until no more apply, then similarly for
Phase~2, and finally for Phase~3. Phase~1 performs let-insertion,
naming subterms that are not values.
%% along the lines of a translation to A-normal form
%% \citep{a-normal-form} or reductions (let.1) and (let.2) in Moggi's
%% metalanguage for monads \citep{moggi}.
Phase~2 performs standard $\beta$ and commuting reductions, and is the
only phase that is crucial for obtaining normal forms that satisfy the
subformula property. Phase~3 ``garbage collects'' unused terms, as in
the call-by-need lambda calculus \citep{call-by-need}. Phase~3 may be
omitted if the intended semantics of the target language is
call-by-value rather than call-by-need.

Every term has a normal form.
\begin{proposition}[Strong normalisation]
Each of the reduction relations $\rewrite{i}$ is strongly
normalising: all $\rewrite{i}$ reduction sequences on well-typed
terms are finite.
\end{proposition}
The only non-trivial proof is for $\rewrite{2}$, which can be proved
via a standard reducibility argument (see, for example,
\cite{Lindley07}). If the target language includes general recursion,
normalisation should treat the fixpoint operator as an uninterpreted
constant.

The grammar of Figure~\ref{fig:nf} characterises normal forms
precisely.
\begin{proposition}[Normal form syntax]
\label{prop_normal}
An expression $N$ matches the syntax of normal forms in
Figure~\ref{fig:nf} if and only if it is in normal form with regard to
the reduction rules of Figure~\ref{fig:norm}.
\end{proposition}

The \emph{subformulas} of a type are the type itself and its
components; for instance, the subformulas of $A \to B$ are $A \to B$
itself and the subformulas of $A$ and $B$. The \emph{proper
  subformulas} of a type are all its subformulas other than the type
itself.  Terms in normal form satisfy the subformula property.

\begin{proposition}[Subformula property]
\label{prop_subformula}
If $\Gamma \vdash M:A$ and the normal form of $M$ is $N$ by the
reduction rules of Figure~\ref{fig:norm}, then $\Gamma \vdash N:A$ and
every subterm of $N$ has a type that is either a subformula of $A$ or
a subformula of a type in $\Gamma$.  Further, every subterm other than
$N$ itself and free variables of $N$ has a type that is a proper
subformula of $A$ or a proper subformula of a type in $\Gamma$.
\end{proposition}




