\documentclass[11p,a4paper]{article}
\usepackage[round]{natbib}
\usepackage{stmaryrd}
\usepackage{proof}
\usepackage{amsmath}
\usepackage{amssymb}
\usepackage{color}
\usepackage{mathtools}
\usepackage[pdfauthor={Shayan Najd},pdftitle={Modular Lazy Normaliser},pagebackref=true,pdftex]{hyperref}
\usepackage{titling}

\newcommand{\incolor}[1]{#1}    % Use to typeset in color
%\newcommand{\incolor}[1]{}     % Use to typeset in black and white
 
% color framework

\newcommand{\judgecolor}{}
\newcommand{\typecolor}{}
\newcommand{\termcolor}{}
\newcommand{\Typecolor}{}
\newcommand{\Termcolor}{}

\newcommand{\uncolored}{
  \incolor{
    \renewcommand{\judgecolor}{}
    \renewcommand{\typecolor}{}
    \renewcommand{\termcolor}{}
    \renewcommand{\Typecolor}{}
    \renewcommand{\Termcolor}{}
  }
}

\newcommand{\colored}{
  \incolor{
    \renewcommand{\judgecolor}{\color{black}}
    \renewcommand{\typecolor}{\color{blue}}
    \renewcommand{\termcolor}{\color{red}}
    \renewcommand{\Typecolor}{\color{cyan}}
    \renewcommand{\Termcolor}{\color{magenta}}
  }
}

\newcommand{\jd}[1]{{\judgecolor #1}}
\newcommand{\tp}[1]{{\typecolor #1}}
\newcommand{\tm}[1]{{\termcolor #1}}
\newcommand{\Tp}[1]{{\Typecolor #1}}
\newcommand{\Tm}[1]{{\Termcolor #1}}

\newcommand{\tmof}[1]{\tm{#1:{}}}
\newcommand{\Tmof}[1]{\Tm{#1:{}}}
\newcommand{\tmsem}[1]{\tm{\sem{\Tm{#1}}}}
\newcommand{\tpsem}[1]{\tp{\sem{\Tp{#1}}}}
\newcommand{\tmSem}[1]{{\termcolor \left\llbracket
                        \judgecolor #1
                        \termcolor \right\rrbracket}}


%%
\newenvironment{note}{\begin{trivlist} \item \textbf{Note. }}{\end{trivlist}}
\newcommand{\todo}[1]{\textbf{#1}}
\newcommand{\etal}{\emph{et~al.}}
\newcommand{\inference}[3]{\infer[\mathsf{#2}]{#3}{#1}}
\newcommand{\Inference}[3]{\infer=[\mathsf{#2}]{#3}{#1}}
\newcommand{\spacer}{\quad\quad\quad\quad}
\newcommand{\filler}{\hspace{0.9\textwidth}}
\newcommand{\tighten}{\vspace{-1ex}}
\newtheorem{theorem}{Theorem}
\newtheorem{lemma}[theorem]{Lemma}
\newtheorem{corollary}[theorem]{Corollary}
\newtheorem{proposition}[theorem]{Proposition}
\newtheorem{definition}[theorem]{Definition}
\newcommand{\hole}{[]}

%% Types
\newcommand{\typvar}[1]{#1}
\newcommand{\typunt}{\boldsymbol{\top}}
\newcommand{\typarr}[2]{#1\boldsymbol{\rightarrow}#2}
\newcommand{\typprd}[2]{#1\boldsymbol{\times}#2}
\newcommand{\typsum}[2]{#1\boldsymbol{+}#2}
\newcommand{\typrec}[2]{\boldsymbol{\mu}#1\boldsymbol{.}#2}

%% Expressions
\newcommand{\expvar}[1]{#1}
\newcommand{\expunt}{\boldsymbol{\top}}
\newcommand{\expabs}[3]{\boldsymbol{\lambda}#1\boldsymbol{\dblcolon}#2\boldsymbol{.}\ #3}
\newcommand{\expapp}[2]{#1\ #2}
\newcommand{\expshr}[3]{\mathbf{Share}\ #1\boldsymbol{=}#2\ \mathbf{in}\ #3} 
\newcommand{\expprd}[2]{\boldsymbol{(}#1\ \boldsymbol{,}\ #2\boldsymbol{)}}
\newcommand{\expfst}[1]{\mathbf{Fst}\ #1}
\newcommand{\expsnd}[1]{\mathbf{Snd}\ #1}
\newcommand{\explft}[2]{\mathbf{Inl}\ #1\ #2}
\newcommand{\exprgt}[2]{\mathbf{Inr}\ #1\ #2}
\newcommand{\expcas}[5]{\mathbf{Case}\ #1\ \mathbf{of}\ \mathbf{Inl}\ #2.\ #3\ \boldsymbol{|}\ \mathbf{Inr}\ #4.\ #5}

%% Environment
\newcommand{\envemp}{\tp{\boldsymbol{\emptyset}}}
\newcommand{\envcon}[2]{\tp{#1,}\ #2}
\newcommand{\env}{\tp{\Gamma}}
\newcommand{\typing}[2]{\tm{#1:\ }\tp{#2}}

\newcommand{\typenvcon}[2]{\tp{\Gamma,}\ \typing{#1}{#2}}
\newcommand{\sbs}[3]{#1[#2:=#3]}
\newcommand{\fv}[1]{FV\ (#1)}

\newcommand{\reduce}[2]{#1 \longrightarrow^* #2} 
\newcommand{\valuep}[1]{Value\ (#1)}

%%%%%%%%%%%%%%%%%%%%%%%%%%%%%%%%%%%%% 
\setlength{\belowcaptionskip}{0pt}
\setlength{\abovecaptionskip}{0pt}
\setlength{\droptitle}{0pt} 
\addtolength{\droptitle}{-60pt}
\addtolength{\textheight}{50pt}
\addtolength{\textwidth}{125pt}
\setlength{\oddsidemargin}{0pt}
\setlength{\evensidemargin}{0pt}
%%\setlength{\marginparsep}{0pt}
%%\setlength{\marginparwidth}{0pt}

\begin{document}
\title{Modular Lazy Normalisation}
\author{Shayan Najd}
\maketitle 
\section*{Introduction}
In this article, I sketch design of a ``modular lazy normalisation''
algorithm.

The normalisation algorithm takes as input an expression and returns a
``semantically equivalent'' expression in ``normal form''. In
particular, we are interested in a normal form where all
$\beta$-redexes, and $\delta$-redexes corresponding to data
desctructors are eliminated.

Normalisation algorithm is ``lazy'' in the sense that elimination of
redexes does not duplicate expressions; it preserves sharing.

Normalisation algorithm is ``modular'' in the sense that the same
algorithm can be applied to a family of languages. Members of
the mentioned family of languages differ only in their set of built-in
functions and built-in types.

\section*{Step I -- No Polymorphism}
Variables, denoted by the nonterminal $X$, are distinct set of
names. Metavariables $X_s$, for arbitrary subscript $s$, range over
variables.

Types, denoted by the nonterminal $T$, are \textbf{closed} terms
defined by the BNF grammar in Figure~\ref{fig:typ}.

\begin{figure*}[h]
\[\uncolored
\begin{array}{llll}
\tp{T} & \mathbin{\ ::=\ } 
%% & \tp{\typvar{X}}        & \textrm{variable}     \\
%% & \ |\ \ 
         & \tp{\typunt}       & \textrm{unit}         \\
& \ |\ \ & \tp{\typarr{T}{T}} & \textrm{abstraction}  \\
& \ |\ \ & \tp{\typprd{T}{T}} & \textrm{product}      \\
& \ |\ \ & \tp{\typsum{T}{T}} & \textrm{sum}       %% \\
%% & \ |\ \ & \tp{\typrec{X}{T}} & \textrm{recursion} 
\end{array}
\]\caption{Types}
\label{fig:typ}
\end{figure*}

Metavariables $T_s$, for arbitrary subscript $s$, range over
types. Types correspond to formule in minimal propositional logic.

%% and we restrict them to occure strictly
%% positive:
%% \todo{definition of strictly positive here}  
%% With exception of the recursion type, 
%% The recursion type is included to be able
%% to encode a wide range of datatypes which are otherwise built-in in
%% the language; every monomorphic datatype can be viewed as a recursive
%% sums of products. This general encoding enables us to provide
%% generic normalisation rules for data destructors. The generic rules
%% need to only consider three data destructors: two destructors for
%% products and one for sums. Since all data destructors are a combination
%% of the mentioned three data destructors, their corresponding
%% $\delta$-redexes are combinations of $\delta$-redexes corresponding to
%% the three data destructors. Therefore, by cooking in a set of
%% normalisation rules to eliminate $\delta$-redexes corresponding to the
%% three data destructors, we gaurantee removal of all $\delta$-redexes
%% corresponding to data destructors; there is no need to provide and
%% build in a separate set of $\delta-$reduction rules per each datatype.
 
Expressions, denoted by the nonterminal $E$, are \textbf{open} terms
defined by the BNF grammar in Figure~\ref{fig:exp}.

\begin{figure*}[h]
\[\uncolored
\begin{array}{llll}
\tm{E} & \mathbin{\ ::=\ } 
         & \tm{\expvar{X}}       & \textrm{variable}          \\
& \ |\ \ & \tm{\expunt}          & \textrm{unit}              \\
& \ |\ \ & \tm{\expabs{X}{T}{E}} & \textrm{abstraction}       \\
& \ |\ \ & \tm{\expapp{E}{E}}    & \textrm{application}       \\
& \ |\ \ & \tm{\expshr{X}{E}{E}} & \textrm{sharing}           \\
& \ |\ \ & \tm{\expprd{E}{E}}    & \textrm{product}           \\
& \ |\ \ & \tm{\expfst{E}}       & \textrm{projection-first}  \\
& \ |\ \ & \tm{\expsnd{E}}       & \textrm{projection-second} \\ 
& \ |\ \ & \tm{\explft{E}{T}}    & \textrm{injection-left}    \\
& \ |\ \ & \tm{\exprgt{T}{E}}    & \textrm{injection-right}   \\
& \ |\ \ & \tm{\expcas{E}{X}{E}{X}{E}} & \textrm{case}           
\end{array}
\]
\caption{Expressions}
\label{fig:exp}
\end{figure*}
 
Metavariables $E_s$, for arbitrary subscript $s$, range over
expressions. Expressions correspond to proofs in minimal propositional
logic.

Subsititution is defined as usual and it is denoted as
$\sbs{E_x}{X_s}{E_s}$ for subtitution of the expression $E_s$ for the
variable $X_s$ in the expression $E_x$. Set of free variables in an
expression is calculated as usual and it is denote as $\fv{E_s}$ for
set of free variables in $E_s$.
% We follow Barendregt convension such that we assume variables in a
% term are all distinct. 
Typing, normalisation, and evaluation environments are all maps from
variable names. The metavariable $\Gamma$ ranges over environments.
The inference rules are equivalent up to exchange, contraction, and
weakening operations.

Typing relations, denoted as $:$ infix operator, are defined by
the inference rules in Figure~\ref{fig:typing}.

\begin{figure*}[h]
\[\colored
\begin{array}{ll}
\fbox{$\env \vdash \typing{E}{T}$}
\\~\\
\inference
{}
{variable}
{
  \typenvcon{X_v}{T_v} \vdash \typing{X_v}{T_v}
}

&

\inference
{}
{unit}
{
   \env \vdash \typing{\expunt}{\typunt}
}
 
\\~\\

\inference
{
  \typenvcon{X_a}{T_a} \vdash \typing{E_b}{T_b} 
}
{abstraction}
{
  \env \vdash \typing{\expabs{X_a}{T_a}{E_b}}{\typarr{T_a}{T_b}}
}
 
&

\inference
{
  \env \vdash \typing{E_f}{\typarr{T_a}{T_b}} 
& \env \vdash \typing{E_a}{T_a} 
}
{application}
{
  \env \vdash \typing{\expapp{E_f}{E_a}}{T_b} 
}

\\~\\ 

\inference
{
  \env \vdash \typing{E_s}{T_s}
  &
  \typenvcon{X_s}{T_s} \vdash \typing{E_b}{T_b}
}
{share}
{
  \env \vdash \typing{\expshr{X_s}{E_s}{E_b}}{T_b}
}
 
&
 
\inference
{
  \env \vdash \typing{E_l}{T_l}
  & 
  \env \vdash \typing{E_r}{T_r} 
}
{product}
{
  \env \vdash \typing{\expprd{E_l}{E_r}}{\typprd{T_l}{T_r}} 
}
 
\\~\\ 

\inference
{
  \env \vdash \typing{E_p}{\typprd{T_l}{T_r}} 
}
{projection-first}
{
  \env \vdash \typing{\expfst{E_p}}{T_l}
}
 
&

\inference
{
  \env \vdash \typing{E_p}{\typprd{T_l}{T_r}} 
}
{projection-second}
{
  \env \vdash \typing{\expsnd{E_p}}{T_r}
}

\\~\\

\inference
{
  \env \vdash \typing{E_l}{T_l}
}
{injection-left}
{
  \env \vdash \typing{\explft{E_l}{T_r}}{\typsum{T_l}{T_r}}
}
 
&
\inference
{
  \env \vdash \typing{E_r}{T_r} 
}
{injection-right}
{
  \env \vdash \typing{\exprgt{T_l}{E_r}}{\typsum{T_l}{T_r}}
}
 
\\~\\

\inference
{
  \env \vdash \typing{E_s}{\typsum{T_l}{T_r}} 
& 
  \typenvcon{X_l}{T_l} \vdash \typing{E_l}{T_c}  
& 
  \typenvcon{X_r}{T_r} \vdash \typing{E_r}{T_c}
}
{case}
{
  \env \vdash \typing{\expcas{E_s}{X_l}{E_l}{X_r}{E_r}}{T_c} 
} 
\end{array}
\]
\caption{Typing Rules}
\label{fig:typing}
\end{figure*}

Typing rules are syntax-directed and algorithmic. The consistency of
the type system can be established by its correspondence (under
Curry-Howard) to minimal logic.\\

Values are expressions that do not compute and they are identified by
the predicate defined in Figure~\ref{fig:val}.

\begin{figure*}[h]
\[\uncolored
\begin{array}{lll}

\fbox{$\valuep{E}$}

\\~\\

\inference
{}
{variable}
{
  \valuep{\expvar{X_v}}
}

 &

\inference
{}
{unit}
{
  \valuep{\expunt}
}

&

\inference
{
   \valuep{E_b}
}
{abstraction}
{
  \valuep{\expabs{X_a}{T_a}{E_b}}
}

\\~\\

\inference
{
  \valuep{E_f} 
  &
  \valuep{E_s} 
}
{product}
{
  \valuep{\expprd{E_f}{E_s}}
}
 
&

\inference
{ 
  \valuep{E_l} 
}
{injection-left}
{
  \valuep{\explft{E_l}{T_r}}
}

&

\inference
{ 
  \valuep{E_r} 
}
{injection-right}
{
  \valuep{\exprgt{T_l}{E_r}}
}

\end{array}
\]
\caption{Values}
\label{fig:val}
\end{figure*}
  
%% Evaluation frames, denoted as $\Tp{F_x}$, are defined by the
%% following grammar:

%% \begin{figure*}
%% \[\uncolored
%% \begin{array}{llll}
%% \Tp{F} & \mathbin{\ ::=\ } & \Tp{\expapp{\hole}{E}} & \textrm{application-fun.}     \\
%% & \ |\ \ & \Tp{\expapp{V}{\hole}}                   & \textrm{application-arg.}     \\
%% & \ |\ \ & \Tp{\expshr{X}{\hole}{E}}                & \textrm{sharing}              \\
%% & \ |\ \ & \Tp{\expprd{\hole}{E}}                   & \textrm{product-first}        \\
%% & \ |\ \ & \Tp{\expprd{V}{\hole}}                   & \textrm{product-second}       \\
%% & \ |\ \ & \Tp{\expfst{\hole}}                      & \textrm{projection-first}     \\
%% & \ |\ \ & \Tp{\expsnd{\hole}}                      & \textrm{projection-second}    \\
%% & \ |\ \ & \Tp{\explft{\hole}{T}}                   & \textrm{injection-left}       \\
%% & \ |\ \ & \Tp{\exprgt{T}{\hole}}                   & \textrm{injection-right}      \\
%% & \ |\ \ & \Tp{\expcas{\hole}{E}{E}}                & \textrm{case}                 \\
%% \end{array}
%% \]
%% \caption{Evaluation Frame}
%% \label{fig:evlfrm}
%% \end{figure*}

Reduction rules, denoted by $\longrightarrow$ infix operator, are
defined in Figure~\ref{fig:red}.Reflexive transivtive closure of the
reduction relation $\longrightarrow$ is denoted as
$\longrightarrow^*$.
   
\begin{figure*}[h]
\[\uncolored
\begin{array}{lcll}
\fbox{$E \longrightarrow E$} \\~\\

\expapp{(\expabs{X_a}{T_a}{E_b})}{E_a} & \longrightarrow 
& \expshr{X_a}{E_a}{E_b} & \\

\expapp{(\expshr{X_l}{E_l}{E_b})}{E_a} & \longrightarrow 
& \expshr{X_l}{E_l}{\expapp{E_b}{E_a}} & if\ X_l \notin \fv{E_a}\\

\expapp{E_a}{(\expshr{X_l}{E_l}{E_b}}) & \longrightarrow 
& \expshr{X_l}{E_l}{\expapp{E_a}{E_b}} & if\ X_l \notin \fv{E_a}\\

\expapp{E_a}{(\expcas{E_c}{X_l}{E_l}{X_r}{E_r}}) & \longrightarrow 
& \mathbf{Case}\ E_c\ \mathbf{of}\\& & \ \ \ \mathbf{Inl}\ X_l.\ (\expapp{E_a}E_l)\ \boldsymbol{|}\ \mathbf{Inr}\ X_r.\ (\expapp{E_a}E_r) 
& \\

\expshr{X_l}{E_l}{X_l}                             & \longrightarrow 
& E_l & \\

\expshr{X_l}{E_l}{E_b}                             & \longrightarrow 
& \sbs{E_b}{X_l}{E_l} & if\ \valuep{E_l} \\

\expshr{X_l}{E_l}{E_b}                             & \longrightarrow 
& E_b                                             & if\ \neg \valuep{E_l}\ \wedge\\ & & &\ \ \ X_l \notin \fv{E_b} \\

\expshr{X_l}{(\expshr{X_{ll}}{E_{ll}}{E_{lb}})}{E_b} & \longrightarrow 
& \expshr{X_{ll}}{E_{ll}}{\\&&\expshr{X_l}{E_{lb}}{E_b}} & if\ X_{ll} \notin \fv{E_b}\\

%%\expprd{E_f}{E_s}                                  & \longrightarrow 
%%& \expshr{X_l}{E_f}{\expprd{X_l}{E_s}}             & if\ \neg \valuep{E_f} \\       
%%
%%\expprd{E_f}{E_s}                                  & \longrightarrow 
%%& \expshr{X_l}{E_s}{\expprd{E_f}{X_l}}             & if\ \valuep{E_f}\ and\ \neg \valuep{E_s} \\       

\expfst{\expprd{E_f}{E_s}}                         & \longrightarrow 
& E_f                                              & \\

\expfst{(\expshr{X_l}{E_l}{E_b})}                   & \longrightarrow 
& \expshr{X_l}{E_l}{\expfst E_b}                    & \\

\expfst{(\expcas{E_c}{X_l}{E_l}{X_r}{E_r})}          & \longrightarrow 
& \mathbf{Case}\ E_c\ \mathbf{of}\\& & \ \ \ \mathbf{Inl}\ X_l.\ (\expfst{E_l})\ \boldsymbol{|}\ \mathbf{Inr}\ X_r.\ (\expfst{E_r}) 
& \\

\expsnd{\expprd{E_f}{E_s}}                          & \longrightarrow 
& E_s                                               & \\

\expsnd{(\expshr{X_l}{E_l}{E_b})}                    & \longrightarrow 
& \expshr{X_l}{E_l}{\expsnd E_b}                     & \\

\expsnd{(\expcas{E_c}{X_l}{E_l}{X_r}{E_r})}          & \longrightarrow 
& \mathbf{Case}\ E_c\ \mathbf{of}\\& & \ \ \ \mathbf{Inl}\ X_l.\ (\expsnd{E_l})\ \boldsymbol{|}\ \mathbf{Inr}\ X_r.\ (\expsnd{E_r}) 
& \\

%%\explft{E_l}{T_r}                                  & \longrightarrow 
%%& \expshr{X_l}{E_l}{\explft{X_l}{T_r}}              & if\ \neg \valuep{E_l}\\
%%
%%\exprgt{T_l}{E_r}                                  & \longrightarrow 
%%& \expshr{X_l}{E_r}{\exprgt{T_l}{X_l}}              & if\ \neg \valuep{E_r}\\
 
\mathbf{Case}\ (\explft{E_c}{T_c})\ \mathbf{of}\\\ \  \mathbf{Inl}\ X_l.\ E_l\ \boldsymbol{|}\ \mathbf{Inr}\ X_r.\ E_r 
& \longrightarrow 
& \expshr{X_l}{E_c}{E_l}                             & \\
 
\mathbf{Case}\ (\exprgt{T_c}{E_c})\ \mathbf{of}\\\ \  \mathbf{Inl}\ X_l.\ E_l\ \boldsymbol{|}\ \mathbf{Inr}\ X_r.\ E_r 
& \longrightarrow 
& \expshr{X_r}{E_c}{E_r}                             & \\

\mathbf{Case}\ (\expshr{X_{cl}}{E_{cl}}{E_b})\ \mathbf{of}\\\ \  \mathbf{Inl}\ X_l.\ E_l\ \boldsymbol{|}\ \mathbf{Inr}\ X_r.\ E_r 
& \longrightarrow 
& \expshr{X_{cl}}{E_{cl}}{\\&&(\expcas{E_b}{X_l}{E_l}{X_r}{E_r})}       & if\ X_l \notin \fv{E_l}\ \wedge \\ & & &\ \ \ X_l \notin \fv{E_r} 
 

\end{array}
\]
\caption{Reduction Rules}
\label{fig:red}
\end{figure*} 

\begin{definition}[Normal Form]\ \\
An expression $E_s$ is in normal form, if and only if, it cannot reduce any further:

$ Norm(E_s) \Leftrightarrow \nexists E_t.\ E_s \longrightarrow E_t$

\end{definition}

\begin{theorem}[Preservation]\ \\
Reduction preserves the type of the source expression:

$(\env \vdash \typing{E_s}{T_s})\wedge(E_s \longrightarrow E_t)\Rightarrow \env \vdash \typing{E_t}{T_s}$
\end{theorem}

\begin{theorem}[Original Subformula Property]\ \\
If an expression $E_s$ reduces to the expression $E_t$ which is in
normal form, then each subexpression of $E_t$ has a type that is a
subterm of type of $E_t$ or a type in the environment:

$$(\Gamma_s \vdash \typing{E_s}{T_s}) \wedge (\reduce{E_s}{E_t}) \wedge (Norm(E_t))
   \Rightarrow
   \forall\ (\Gamma_i \vdash \typing{E_i}{T_i}) \in E_t.\ (T_i \in T_s) 
            \vee 
            (\exists\ T_e \in \Gamma_s. T_i \in T_e)$$

TODO: Define what $\in$ means for types and expressions
\end{theorem}


 
\end{document}
