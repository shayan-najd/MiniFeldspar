\documentclass[11p,a4paper]{article}
\usepackage[round]{natbib}
\usepackage{stmaryrd}
\usepackage{proof}
\usepackage{amsmath}
\usepackage{amssymb}
\usepackage{color}
\usepackage{mathtools}
\usepackage[pdfauthor={Shayan Najd},pdftitle={Modular Lazy Normaliser},pagebackref=true,pdftex]{hyperref}
\usepackage{titling}

\newcommand{\incolor}[1]{#1}    % Use to typeset in color
%\newcommand{\incolor}[1]{}     % Use to typeset in black and white
 
% color framework

\newcommand{\judgecolor}{}
\newcommand{\typecolor}{}
\newcommand{\termcolor}{}
\newcommand{\Typecolor}{}
\newcommand{\Termcolor}{}

\newcommand{\uncolored}{
  \incolor{
    \renewcommand{\judgecolor}{}
    \renewcommand{\typecolor}{}
    \renewcommand{\termcolor}{}
    \renewcommand{\Typecolor}{}
    \renewcommand{\Termcolor}{}
  }
}

\newcommand{\colored}{
  \incolor{
    \renewcommand{\judgecolor}{\color{black}}
    \renewcommand{\typecolor}{\color{blue}}
    \renewcommand{\termcolor}{\color{red}}
    \renewcommand{\Typecolor}{\color{cyan}}
    \renewcommand{\Termcolor}{\color{magenta}}
  }
}

\newcommand{\jd}[1]{{\judgecolor #1}}
\newcommand{\tp}[1]{{\typecolor #1}}
\newcommand{\tm}[1]{{\termcolor #1}}
\newcommand{\Tp}[1]{{\Typecolor #1}}
\newcommand{\Tm}[1]{{\Termcolor #1}}

\newcommand{\tmof}[1]{\tm{#1:{}}}
\newcommand{\Tmof}[1]{\Tm{#1:{}}}
\newcommand{\tmsem}[1]{\tm{\sem{\Tm{#1}}}}
\newcommand{\tpsem}[1]{\tp{\sem{\Tp{#1}}}}
\newcommand{\tmSem}[1]{{\termcolor \left\llbracket
                        \judgecolor #1
                        \termcolor \right\rrbracket}}


%%
\newenvironment{note}{\begin{trivlist} \item \textbf{Note. }}{\end{trivlist}}
\newcommand{\todo}[1]{\textbf{#1}}
\newcommand{\etal}{\emph{et~al.}}
\newcommand{\inference}[3]{\infer[\mathsf{#2}]{#3}{#1}}
\newcommand{\Inference}[3]{\infer=[\mathsf{#2}]{#3}{#1}}
\newcommand{\spacer}{\quad\quad\quad\quad}
\newcommand{\filler}{\hspace{0.9\textwidth}}
\newcommand{\tighten}{\vspace{-1ex}}
\newtheorem{theorem}{Theorem}
\newtheorem{lemma}[theorem]{Lemma}
\newtheorem{corollary}[theorem]{Corollary}
\newtheorem{proposition}[theorem]{Proposition}
\newtheorem{definition}[theorem]{Definition}
\newcommand{\hole}{[]}
\newcommand{\intro}{\mathcal{I}}
\newcommand{\elim}{\mathcal{E}}

%% Types
\newcommand{\typvar}[1]{#1}
\newcommand{\typunt}{\boldsymbol{\top}}
\newcommand{\typarr}[2]{#1\boldsymbol{\rightarrow}#2}
\newcommand{\typprd}[2]{#1\boldsymbol{\times}#2}
\newcommand{\typsum}[2]{#1\boldsymbol{+}#2}
\newcommand{\typrec}[2]{\boldsymbol{\mu}#1\boldsymbol{.}#2}

%% Expressions
\newcommand{\expvar}[1]{#1}
\newcommand{\expunt}{\boldsymbol{\top}}
\newcommand{\expabs}[3]{\boldsymbol{\lambda}#1\boldsymbol{\colon}#2\boldsymbol{.}\ #3}
\newcommand{\expapp}[2]{#1\ #2}
\newcommand{\expshr}[3]{\mathbf{let}\ #1\boldsymbol{=}#2\ \mathbf{in}\ #3} 
\newcommand{\expprd}[2]{\boldsymbol{(}#1\ \boldsymbol{,}\ #2\boldsymbol{)}}
\newcommand{\expfst}[1]{\mathbf{fst}\ #1}
\newcommand{\expsnd}[1]{\mathbf{snd}\ #1}
\newcommand{\explft}[2]{\mathbf{inl}\ #1\ #2}
\newcommand{\exprgt}[2]{\mathbf{inr}\ #1\ #2}
\newcommand{\expcas}[5]{\mathbf{case}\ #1\ \mathbf{of}\ \mathbf{inl}\ #2.\ #3\ \boldsymbol{|}\ \mathbf{inr}\ #4.\ #5}

%% Environment
\newcommand{\envemp}{\tp{\boldsymbol{\emptyset}}}
\newcommand{\envcon}[2]{\tp{#1,}\ #2}
\newcommand{\env}{\tp{\Gamma}}
\newcommand{\typing}[2]{\tm{#1:\ }\tp{#2}}

\newcommand{\typenvcon}[2]{\tp{\Gamma,}\ \typing{#1}{#2}}
\newcommand{\sbs}[3]{#1[#2:=#3]}
\newcommand{\fv}[1]{FV\ (#1)}

\newcommand{\reduce}[2]{#1 \longrightarrow^* #2} 
\newcommand{\valuep}[1]{Value\ (#1)}

%%%%%%%%%%%%%%%%%%%%%%%%%%%%%%%%%%%%% 
\setlength{\belowcaptionskip}{0pt}
\setlength{\abovecaptionskip}{0pt}
\setlength{\droptitle}{0pt} 
\addtolength{\droptitle}{-60pt}
\addtolength{\textheight}{50pt}
\addtolength{\textwidth}{125pt}
\setlength{\oddsidemargin}{0pt}
\setlength{\evensidemargin}{0pt}
%%\setlength{\marginparsep}{0pt}
%%\setlength{\marginparwidth}{0pt}

\begin{document}
\title{Modular Lazy Normalisation}
\author{Shayan Najd}
\maketitle 
\section*{Introduction}
In this article, I sketch design of a ``modular lazy normalisation''
algorithm.

The normalisation algorithm takes as input an expression and returns a
``semantically equivalent'' expression in ``normal form''. In
particular, we are interested in a normal form where all
$\beta$-redexes, and $\delta$-redexes corresponding to data
desctructors are eliminated.

Normalisation algorithm is ``lazy'' in the sense that elimination of
redexes does not duplicate expressions; it preserves sharing.

Normalisation algorithm is ``modular'' in the sense that the same
algorithm can be applied to a family of languages. Members of
the mentioned family of languages differ only in their set of built-in
functions and built-in types.

\section*{Step I -- No Polymorphism}
Variables are distinct set of names. Metavariables $s$, $t$, $x$, $y$, and $z$
range over variables.

Types are \textbf{closed} terms defined by the BNF grammar in
Figure~\ref{fig:typ}.

\begin{figure*}[h]
\[\uncolored
\begin{array}{llll}
\tp{A,B,C} & \mathbin{\ ::=\ } 
%% & \tp{\typvar{X}}        & \textrm{variable}     \\
%% & \ |\ \ 
         & \tp{\typunt}       & \textrm{unit}         \\
& \ |\ \ & \tp{\typarr{A}{B}} & \textrm{abstraction}  \\
& \ |\ \ & \tp{\typprd{A}{B}} & \textrm{product}      \\
& \ |\ \ & \tp{\typsum{A}{B}} & \textrm{sum}       %% \\
%% & \ |\ \ & \tp{\typrec{X}{T}} & \textrm{recursion} 
\end{array}
\]\caption{Types}
\label{fig:typ}
\end{figure*}

Metavariables $A$, $B$, and $C$ range over types. Types correspond to
formule in minimal propositional logic.

%% and we restrict them to occure strictly
%% positive:
%% \todo{definition of strictly positive here}  
%% With exception of the recursion type, 
%% The recursion type is included to be able
%% to encode a wide range of datatypes which are otherwise built-in in
%% the language; every monomorphic datatype can be viewed as a recursive
%% sums of products. This general encoding enables us to provide
%% generic normalisation rules for data destructors. The generic rules
%% need to only consider three data destructors: two destructors for
%% products and one for sums. Since all data destructors are a combination
%% of the mentioned three data destructors, their corresponding
%% $\delta$-redexes are combinations of $\delta$-redexes corresponding to
%% the three data destructors. Therefore, by cooking in a set of
%% normalisation rules to eliminate $\delta$-redexes corresponding to the
%% three data destructors, we gaurantee removal of all $\delta$-redexes
%% corresponding to data destructors; there is no need to provide and
%% build in a separate set of $\delta-$reduction rules per each datatype.
 
Expressions are \textbf{open} terms defined by the BNF grammar in
Figure~\ref{fig:exp}.

\begin{figure*}[h]
\[\uncolored
\begin{array}{llll}
\tm{J,K,L,M,N} & \mathbin{\ ::=\ } 
         & \tm{\expvar{x}}       & \textrm{variable}          \\
& \ |\ \ & \tm{\expunt}          & \textrm{unit}              \\
& \ |\ \ & \tm{\expabs{x}{A}{N}} & \textrm{abstraction}       \\
& \ |\ \ & \tm{\expapp{L}{M}}    & \textrm{application}       \\
& \ |\ \ & \tm{\expshr{x}{M}{N}} & \textrm{sharing}           \\
& \ |\ \ & \tm{\expprd{M}{N}}    & \textrm{product}           \\
& \ |\ \ & \tm{\expfst{L}}       & \textrm{projection-first}  \\
& \ |\ \ & \tm{\expsnd{L}}       & \textrm{projection-second} \\ 
& \ |\ \ & \tm{\explft{M}{B}}    & \textrm{injection-left}    \\
& \ |\ \ & \tm{\exprgt{A}{N}}    & \textrm{injection-right}   \\
& \ |\ \ & \tm{\expcas{L}{x}{M}{y}{N}} & \textrm{case}           
\end{array}
\]
\caption{Expressions}
\label{fig:exp}
\end{figure*}
 
Metavariables $J$, $K$, $L$, $M$, and $N$ range over
expressions. Expressions correspond to proofs in minimal propositional
logic.

Subsititution is defined as usual and it is denoted as
$\sbs{N}{x}{M}$ for subtitution of the expression $M$ for the
variable $x$ in the expression $N$. Set of free variables in an
expression is calculated as usual and it is denote as $\fv{N}$ for
set of free variables in $N$.
% We follow Barendregt convension such that we assume variables in a
% term are all distinct. 
Typing, normalisation, and evaluation environments are all maps from
variable names. The metavariable $\Gamma$ ranges over environments.
The inference rules are equivalent up to exchange, contraction, and
weakening operations.

Typing relations, denoted as $:$ infix operator, are defined by
the inference rules in Figure~\ref{fig:typing}.

\begin{figure*}[h]
\[\colored
\begin{array}{ll}
\fbox{$\env \vdash \typing{M}{A}$}
\\~\\
\inference
{}
{\mathbf{Ax}}
{
  \typenvcon{x}{A} \vdash \typing{x}{A}
}

&

\inference
{}
{\typunt}
{
   \env \vdash \typing{\expunt}{\typunt}
}
 
\\~\\

\inference
{
  \typenvcon{x}{A} \vdash \typing{N}{B} 
}
{{\to}\intro}
{
  \env \vdash \typing{\expabs{x}{A}{N}}{\typarr{A}{B}}
}
 
&

\inference
{
  \env \vdash \typing{L}{\typarr{A}{B}} 
& \env \vdash \typing{M}{A} 
}
{{\to}\elim}
{
  \env \vdash \typing{\expapp{L}{M}}{B} 
}

\\~\\ 

\inference
{
  \env \vdash \typing{M}{A}
  &
  \typenvcon{x}{A} \vdash \typing{N}{B}
}
{\mathbf{let}}
{
  \env \vdash \typing{\expshr{x}{M}{N}}{B}
}
 
&
 
\inference
{
  \env \vdash \typing{M}{A}
  & 
  \env \vdash \typing{N}{B} 
}
{{\times}\intro}
{
  \env \vdash \typing{\expprd{M}{N}}{\typprd{A}{B}} 
}
 
\\~\\ 

\inference
{
  \env \vdash \typing{L}{\typprd{A}{B}} 
}
{{\times}\elim_1}
{
  \env \vdash \typing{\expfst{L}}{A}
}
 
&

\inference
{
  \env \vdash \typing{L}{\typprd{A}{B}} 
}
{{\times}\elim_2}
{
  \env \vdash \typing{\expsnd{L}}{B}
}

\\~\\

\inference
{
  \env \vdash \typing{M}{A}
}
{{+}\intro_1}
{
  \env \vdash \typing{\explft{M}{B}}{\typsum{A}{B}}
}
 
&
\inference
{
  \env \vdash \typing{N}{B} 
}
{{+}\intro_2}
{
  \env \vdash \typing{\exprgt{A}{N}}{\typsum{A}{B}}
}
 
\\~\\

\inference
{
  \env \vdash \typing{L}{\typsum{A}{B}} 
& 
  \typenvcon{x}{A} \vdash \typing{M}{C}  
& 
  \typenvcon{y}{B} \vdash \typing{N}{C}
}
{{+}\elim}
{
  \env \vdash \typing{\expcas{L}{x}{M}{y}{N}}{C} 
} 
\end{array}
\]
\caption{Typing Rules}
\label{fig:typing}
\end{figure*}

Typing rules are syntax-directed and algorithmic. The consistency of
the type system can be established by its correspondence (under
Curry-Howard) to minimal logic.\\

Values are expressions that do not compute and they are identified by
the predicate defined in Figure~\ref{fig:val}.

\begin{figure*}[h]
\[\uncolored
\begin{array}{lll}

\fbox{$\valuep{N}$}

\\~\\

\inference
{}
{variable}
{
  \valuep{\expvar{x}}
}

 &

\inference
{}
{unit}
{
  \valuep{\expunt}
}

&

\inference
{
   \valuep{N}
}
{abstraction}
{
  \valuep{\expabs{x}{A}{N}}
}

\\~\\

\inference
{
  \valuep{M} 
  &
  \valuep{N} 
}
{product}
{
  \valuep{\expprd{M}{N}}
}
 
&

\inference
{ 
  \valuep{M} 
}
{injection-left}
{
  \valuep{\explft{M}{B}}
}

&

\inference
{ 
  \valuep{N} 
}
{injection-right}
{
  \valuep{\exprgt{A}{N}}
}

\end{array}
\]
\caption{Values}
\label{fig:val}
\end{figure*}
  
%% Evaluation frames, denoted as $\Tp{F_x}$, are defined by the
%% following grammar:

%% \begin{figure*}
%% \[\uncolored
%% \begin{array}{llll}
%% \Tp{F} & \mathbin{\ ::=\ } & \Tp{\expapp{\hole}{E}} & \textrm{application-fun.}     \\
%% & \ |\ \ & \Tp{\expapp{V}{\hole}}                   & \textrm{application-arg.}     \\
%% & \ |\ \ & \Tp{\expshr{X}{\hole}{E}}                & \textrm{sharing}              \\
%% & \ |\ \ & \Tp{\expprd{\hole}{E}}                   & \textrm{product-first}        \\
%% & \ |\ \ & \Tp{\expprd{V}{\hole}}                   & \textrm{product-second}       \\
%% & \ |\ \ & \Tp{\expfst{\hole}}                      & \textrm{projection-first}     \\
%% & \ |\ \ & \Tp{\expsnd{\hole}}                      & \textrm{projection-second}    \\
%% & \ |\ \ & \Tp{\explft{\hole}{T}}                   & \textrm{injection-left}       \\
%% & \ |\ \ & \Tp{\exprgt{T}{\hole}}                   & \textrm{injection-right}      \\
%% & \ |\ \ & \Tp{\expcas{\hole}{E}{E}}                & \textrm{case}                 \\
%% \end{array}
%% \]
%% \caption{Evaluation Frame}
%% \label{fig:evlfrm}
%% \end{figure*}

Reduction rules, denoted by $\longrightarrow$ infix operator, are
defined in Figure~\ref{fig:red}.Reflexive transivtive closure of the
reduction relation $\longrightarrow$ is denoted as
$\longrightarrow^*$.
   
\begin{figure*}[h]
\[\uncolored
\begin{array}{lcll}
%% \fbox{$M \longrightarrow N$} \\~\\

\expapp{(\expabs{x}{A}{M})}{N} & \longrightarrow 
& \expshr{x}{N}{M} & \\

\expapp{(\expshr{x}{L}{M})}{N} & \longrightarrow 
& \expshr{x}{L}{\expapp{M}{N}} & if\ x \notin \fv{N}\\

\expapp{L}{(\expshr{x}{M}{N}}) & \longrightarrow 
& \expshr{x}{M}{\expapp{L}{N}} & if\ x \notin \fv{L}\\

\expapp{K}{(\expcas{L}{x}{M}{y}{N}}) & \longrightarrow 
& \mathbf{case}\ L\ \mathbf{of}\\& & \ \ \ \mathbf{inl}\ x.\ (\expapp{K}{M})\ \boldsymbol{|}\ \mathbf{inr}\ y.\ (\expapp{K}{N}) 
& if\ x \notin \fv{K}\ \wedge\\ & & &\ \ \ \ y \notin \fv{K} \\

\expshr{x}{M}{x}                             & \longrightarrow 
& M & \\

\expshr{x}{M}{N}                             & \longrightarrow 
& \sbs{N}{x}{M} & if\ \valuep{M} \\

\expshr{x}{M}{N}                             & \longrightarrow 
& N                                          & if\ x \notin \fv{N} \\

\expshr{z}{(\expshr{x}{M}{N})}{L} & \longrightarrow 
& \expshr{x}{M}{\\&&\expshr{z}{N}{L}} & if\ x \notin \fv{L}\\

%%\expprd{E_f}{E_s}                                  & \longrightarrow 
%%& \expshr{x}{E_f}{\expprd{x}{E_s}}             & if\ \neg \valuep{E_f} \\       
%%
%%\expprd{E_f}{E_s}                                  & \longrightarrow 
%%& \expshr{x}{E_s}{\expprd{E_f}{x}}             & if\ \valuep{E_f}\ and\ \neg \valuep{E_s} \\       

\expfst{\expprd{M}{N}}                         & \longrightarrow 
& M                                              & \\

\expfst{(\expshr{x}{M}{N})}                   & \longrightarrow 
& \expshr{x}{M}{\expfst{N}}                    & \\

\expfst{(\expcas{L}{x}{M}{y}{N})}          & \longrightarrow 
& \mathbf{case}\ L\ \mathbf{of}\\& & \ \ \ \mathbf{inl}\ x.\ (\expfst{M})\ \boldsymbol{|}\ \mathbf{inr}\ y.\ (\expfst{N}) 
& \\

\expsnd{\expprd{M}{N}}                          & \longrightarrow 
& N                                               & \\

\expsnd{(\expshr{x}{M}{N})}                    & \longrightarrow 
& \expshr{x}{M}{\expsnd{N}}                     & \\

\expsnd{(\expcas{L}{x}{M}{y}{N})}          & \longrightarrow 
& \mathbf{case}\ L\ \mathbf{of}\\& & \ \ \ \mathbf{inl}\ x.\ (\expsnd{M})\ \boldsymbol{|}\ \mathbf{inr}\ y.\ (\expsnd{N}) 
& \\

%%\explft{E_l}{T_r}                                  & \longrightarrow 
%%& \expshr{x}{E_l}{\explft{x}{T_r}}              & if\ \neg \valuep{E_l}\\
%%
%%\exprgt{T_l}{E_r}                                  & \longrightarrow 
%%& \expshr{x}{E_r}{\exprgt{T_l}{x}}              & if\ \neg \valuep{E_r}\\
 
\mathbf{case}\ (\explft{L}{B})\ \mathbf{of}\\\ \  \mathbf{inl}\ x.\ M\ \boldsymbol{|}\ \mathbf{inr}\ y.\ N 
& \longrightarrow 
& \expshr{x}{L}{M}                             & \\
 
\mathbf{case}\ (\exprgt{A}{L})\ \mathbf{of}\\\ \  \mathbf{inl}\ x.\ M\ \boldsymbol{|}\ \mathbf{inr}\ y.\ N 
& \longrightarrow 
& \expshr{y}{L}{N}                             & \\

\mathbf{case}\ (\expshr{z}{K}{L})\ \mathbf{of}\\\ \  \mathbf{inl}\ x.\ M\ \boldsymbol{|}\ \mathbf{inr}\ y.\ N 
& \longrightarrow 
& \expshr{z}{K}{\\&&(\expcas{L}{x}{M}{y}{N})}       & if\ z \notin \fv{M}\ \wedge \\ & & &\ \ \ \ z \notin \fv{N}  \\
 
\mathbf{case}\ (\expcas{L}{x}{M}{y}{N})\ \mathbf{of}\\\ \  \mathbf{inl}\ s.\ J\ \boldsymbol{|}\ \mathbf{inr}\ t.\ K 
& \longrightarrow 
& \mathbf{case}\ L \ \mathbf{of}\\ & &\ \ \ \ \mathbf{inl}\ x.\ (\expcas{M}{s}{J}{t}{K})\ \\ & &\ \ \boldsymbol{|}\ \mathbf{inr}\ y.\ (\expcas{N}{s}{J}{t}{K}) 


\end{array}
\]
\caption{Reduction Rules}
\label{fig:red}
\end{figure*} 

\begin{definition}[Normal Form]\ \\
An expression $M$ is in normal form, denoted as $Norm(M)$, if and only
if, does not exist an expression $N$ such that $M \longrightarrow
N$.
\end{definition}

\begin{theorem}[Preservation]\ \\
Reduction preserves the type of the source expression:
 
 if $\env \vdash \typing{M}{A}$ and $M \longrightarrow N$, then 
$\env \vdash \typing{N}{A}$.
\end{theorem}

\begin{theorem}[Subformula Property]\ \\
If $\Gamma_s \vdash \typing{M}{A}$, $\reduce{M}{N}$, and $Norm(N)$;
then for every expression $L$ in the set of subexpressions of $N$ such
that $\Gamma_i \vdash \typing{L}{C}$, $C$ is in the set of subterms of
the type $A$, or in the set of subterms of all types in the
environment $\Gamma_i$.
\end{theorem}
 
\end{document}
